\documentclass[14pt, a4paper]{article}
\usepackage[a4paper,tmargin=2cm, bmargin=2cm, lmargin=3cm, rmargin=2.0cm]{geometry}
\usepackage{booktabs}
\usepackage{anyfontsize}
\usepackage{caption}
\usepackage{subcaption}
\usepackage{subfigure}
\usepackage[vietnam]{babel}
\usepackage{indentfirst}
\usepackage{color}
\usepackage[utf8]{inputenc}
\usepackage{emoji}
\usepackage{amsmath}
\usepackage{amssymb}
\usepackage{amsfonts}
\usepackage{graphics}
\usepackage{graphicx}
% \usepackage{pdfpages}
\usepackage{longtable, makecell, multirow, booktabs}
% \usepackage{caption}
% \usepackage{makecell}
\usepackage{lscape}
\usepackage{pdflscape}
\usepackage{everypage}
\usepackage{titlesec}
\usepackage{tocloft} % Để tùy chỉnh mục lục
\usepackage{fontspec}
\usepackage[thmmarks, amsmath]{ntheorem}%
\usepackage{mdframed}
\usepackage{xeCJK}
\usepackage{array}
\usepackage{arydshln}
\usepackage{pgfgantt}
\usepackage[ruled,vlined,linesnumbered,resetcount,algosection]{algorithm2e}
% \usepackage{algorithm}
% \usepackage{algpseudocode}
% \usepackage{algcompatible}
\usepackage{adjustbox}
\setmainfont{Times New Roman}
\usepackage{hyperref}
\usepackage{tabularx}
\usepackage{float}
\usepackage{pgfgantt}
\usepackage{xcolor}
\usepackage{comment}
%dong khung


% ...existing code...
\usepackage{hyperref}
\hypersetup{
    colorlinks=true,           % Dùng màu chữ thay vì khung
    linkcolor=black,          % Màu link nội bộ (mục lục)
    citecolor=blue,           % Màu citation
    urlcolor=blue,            % Màu URL
    pdfborder={0 0 0},        % Tắt hoàn toàn border
}
% ...existing code...




\usepackage{framed}
\definecolor{barblue}{RGB}{153,204,254}
\definecolor{groupblue}{RGB}{51,102,254}
\definecolor{linkred}{RGB}{165,0,33}

%\renewcommand\sfdefault{phv}
%\renewcommand\mddefault{mc}

% Gây lỗi không in đậm được
% \renewcommand\bfdefault{bc}

\setganttlinklabel{s-s}{START-TO-START}
\setganttlinklabel{f-s}{FINISH-TO-START}
\setganttlinklabel{f-f}{FINISH-TO-FINISH}
\sffamily

%\input setbmp
\usepackage[export]{adjustbox}
% \usepackage{subcaption}

\usepackage{tikz}
\usepackage{pgfplots}
%code 
\usepackage{listings}
\newcommand{\Lpagenumber}{\ifdim\textwidth=\linewidth\else\bgroup
  \dimendef\margin=0 %use \margin instead of \dimen0
  \ifodd\value{page}\margin=\oddsidemargin
  \else\margin=\evensidemargin
  \fi
  \raisebox{\dimexpr -\topmargin-\headheight-\headsep-0.5\linewidth}[0pt][0pt]{%
    \rlap{\hspace{\dimexpr \margin+\textheight+\footskip}%
    \llap{\rotatebox{90}{\thepage}}}}%
\egroup\fi}
\AddEverypageHook{\Lpagenumber}%

% Can le van ban
\usepackage[left=3cm,right=2cm,top=2cm,bottom=2cm]{geometry}
%\usepackage{fancyhdr}

% Numbering for section, subsection, etc.
\renewcommand{\thesection}{{}}
\titlespacing{\section}{2pt}{0pt}{2pt}
\titleformat{\section}{\bfseries\large}{\thesection.}{\hspace0.5em}}{}
\renewcommand{\thesubsection}{\arabic{subsection}}
\titleformat{\subsection}{\bfseries\large}{\thesubsection.}{\hspace{0.5em}}{}

% subsubsbusection
\titleclass{\subsubsubsection}{straight}[\subsection]

\newcounter{subsubsubsection}[subsubsection]
\renewcommand\thesubsubsubsection{\thesubsubsection.\arabic{subsubsubsection}}
\renewcommand\theparagraph{\thesubsubsubsection.\arabic{paragraph}} % optional; useful if paragraphs are to be numbered

\titleformat{\subsubsubsection}
  {\normalfont\normalsize\bfseries}{\thesubsubsubsection}{1em}{}
\titlespacing*{\subsubsubsection}
{0pt}{3.25ex plus 1ex minus .2ex}{1.5ex plus .2ex}

\makeatletter
\renewcommand\paragraph{\@startsection{paragraph}{5}{\z@}%
  {3.25ex \@plus1ex \@minus.2ex}%
  {-1em}%
  {\normalfont\normalsize\bfseries}}
\renewcommand\subparagraph{\@startsection{subparagraph}{6}{\parindent}%
  {3.25ex \@plus1ex \@minus .2ex}%
  {-1em}%
  {\normalfont\normalsize\bfseries}}
\def\toclevel@subsubsubsection{4}
\def\toclevel@paragraph{5}
\def\toclevel@paragraph{6}
\def\l@subsubsubsection{\@dottedtocline{4}{7em}{4em}}
\def\l@paragraph{\@dottedtocline{5}{10em}{5em}}
\def\l@subparagraph{\@dottedtocline{6}{14em}{6em}}
\makeatother

\setcounter{secnumdepth}{4}
\setcounter{tocdepth}{4}


% Numbering for label itemization
\renewcommand{\baselinestretch}{1.3}
\renewcommand{\labelitemi}{$-$}
\renewcommand{\labelitemii}{$+$}

\usepackage{scrextend}
\changefontsizes{13pt}

\renewcommand{\contentsname}{MỤC LỤC}

\renewcommand{\listfigurename}{DANH SÁCH HÌNH VẼ}

\renewcommand{\listtablename}{DANH SÁCH BẢNG}

\renewcommand{\tablename}{}

\renewcommand{\figurename}{}

\let\sectionautorefname\sectionautorefname
% \let\subsectionautorefname\sectionautorefname
% \let\subsubsectionautorefname\sectionautorefname
% \let\subsubsubsectionautorefname\sectionautorefname
\usepackage{textcomp}
\usepackage{listings,cleveref}
%\usepackage{minted}      % (requires -shell-escape)
\usepackage{xcolor}
\usepackage{filecontents}
\theoremheaderfont{\bfseries\upshape}
\theorembodyfont{\normalfont}
\theoremstyle{break}
\theoremseparator{\smallskip}
\newtheorem{defi}{Định nghĩa}[section]
\theoremstyle{break}
\theoremseparator{\smallskip}
\newtheorem{example}[defi]{Ví dụ}
\DeclareMathOperator*{\argmax}{argmax}
\DeclareMathOperator*{\argmin}{argmin}
\newcommand{\source}[1]{\caption*{Nguồn: {#1}} }
\newcommand{\secref}[1]{\autoref{#1}. \nameref{#1}}
\newcommand{\myparagraph}[1]{\paragraph{#1}\mbox{}\\}
\newcommand{\mysubparagraph}[1]{\subparagraph{#1}\mbox{}\\}

\newcommand\vartextvisiblespace[1][.3em]{%
  \mbox{\kern.1em\vrule height.3ex}%
  \vbox{\hrule width#1}%
  \hbox{\vrule height.3ex}
}

\newcolumntype{C}[1]{>{\centering\arraybackslash}m{#1}} 







\begin{document}
 \pagenumbering{gobble}
% \captionsetup{justification=centering}

\captionsetup[figure]{labelformat=simple, labelsep=period}

\captionsetup[table]{labelformat=simple, labelsep=period}

\input{components/title}
%% \section*{THÔNG TIN PROJECT}
% \subsection*{Thư mục nộp bài}
% Đường dẫn: \href{https://drive.google.com/drive/folders/1JTm_curG5sIeihPt1hf0YQC0DFow91e0?usp=sharing}{\textcolor{blue}{Google Drive}}

% \subsection*{Demo}
% Đường dẫn: \href{https://youtu.be/Mjv2eUaNiik}{\textcolor{blue}{Youtube}}

% \subsection*{Cấu trúc thư mục nộp bài trên Google Drive}
% \begin{itemize}
%     \item \textbf{dataset}: Chứa tập dữ liệu sử dụng (ESD).
%     \item \textbf{source}: Mã nguồn chính của bài báo cáo.
%     \begin{itemize}
%         \item \textbf{code\_train}: Code để huấn luyện các mô hình (code cũng đã push lên \href{https://github.com/tiep271001/TriTueNhanTao_Nhom6}{\textcolor{blue}{\underline{github}}}).
%         \item \textbf{speech\_recognition\_serving}: Code triển khai mô hình (code cũng đã push lên \href{https://github.com/tiep271001/TriTueNhanTao_Nhom6}{\textcolor{blue}{\underline{github}}}).
%         \item \textbf{speech\_recognition\_app}: Code của ứng dụng (code cũng đã push lên \href{https://github.com/tiep271001/TriTueNhanTao_Nhom6}{\textcolor{blue}{\underline{github}}}).
%     \end{itemize}
%     \item \textbf{checkpoint}: Chứa checkpoint của các mô hình đã huấn luyện.
%     \item \textbf{Image\_speech\_recognition}: Chứa tất cả hình ảnh trong pptx và báo cáo.
% \end{itemize}

% \subsection*{Manual}
% Các bước để chạy web app:
% \begin{enumerate}
%     \item Truy cập \href{https://0551-2001-ee0-5006-9a10-23d3-e0c-4e83-ce7f.ngrok-free.app/}{\textcolor{blue}{đường dẫn}} đến ứng dụng 
%     \item \textbf{Click to record} để thu âm
%     \item \textbf{Click to stop recording} để kết thúc quá trình thu âm
%     \item \textbf{Send to Server} để nhận kết quả
% \end{enumerate}

% Đường link cập nhật của ứng dụng tại \href{https://drive.google.com/drive/folders/1JTm_curG5sIeihPt1hf0YQC0DFow91e0?usp=sharing}{\textcolor{blue}{Google Drive}}
% \newpage
\tableofcontents
\thispagestyle{empty}
\newpage
\thispagestyle{empty}
% \listoftables
% \newpage
\listoffigures
\newpage
% \setcounter{secnumdepth}{0} % Tắt đánh số section
\listoftables
\clearpage

\setcounter{secnumdepth}{9} % Đảm bảo đánh số cho các cấp thấp hơn

\pagenumbering{arabic}
% \section*{THÔNG TIN PROJECT}
% \subsection*{Thư mục nộp bài}
% Đường dẫn: \href{https://drive.google.com/drive/folders/1JTm_curG5sIeihPt1hf0YQC0DFow91e0?usp=sharing}{\textcolor{blue}{Google Drive}}

% \subsection*{Demo}
% Đường dẫn: \href{https://youtu.be/Mjv2eUaNiik}{\textcolor{blue}{Youtube}}

% \subsection*{Cấu trúc thư mục nộp bài trên Google Drive}
% \begin{itemize}
%     \item \textbf{dataset}: Chứa tập dữ liệu sử dụng (ESD).
%     \item \textbf{source}: Mã nguồn chính của bài báo cáo.
%     \begin{itemize}
%         \item \textbf{code\_train}: Code để huấn luyện các mô hình (code cũng đã push lên \href{https://github.com/tiep271001/TriTueNhanTao_Nhom6}{\textcolor{blue}{\underline{github}}}).
%         \item \textbf{speech\_recognition\_serving}: Code triển khai mô hình (code cũng đã push lên \href{https://github.com/tiep271001/TriTueNhanTao_Nhom6}{\textcolor{blue}{\underline{github}}}).
%         \item \textbf{speech\_recognition\_app}: Code của ứng dụng (code cũng đã push lên \href{https://github.com/tiep271001/TriTueNhanTao_Nhom6}{\textcolor{blue}{\underline{github}}}).
%     \end{itemize}
%     \item \textbf{checkpoint}: Chứa checkpoint của các mô hình đã huấn luyện.
%     \item \textbf{Image\_speech\_recognition}: Chứa tất cả hình ảnh trong pptx và báo cáo.
% \end{itemize}

% \subsection*{Manual}
% Các bước để chạy web app:
% \begin{enumerate}
%     \item Truy cập \href{https://0551-2001-ee0-5006-9a10-23d3-e0c-4e83-ce7f.ngrok-free.app/}{\textcolor{blue}{đường dẫn}} đến ứng dụng 
%     \item \textbf{Click to record} để thu âm
%     \item \textbf{Click to stop recording} để kết thúc quá trình thu âm
%     \item \textbf{Send to Server} để nhận kết quả
% \end{enumerate}

% Đường link cập nhật của ứng dụng tại \href{https://drive.google.com/drive/folders/1JTm_curG5sIeihPt1hf0YQC0DFow91e0?usp=sharing}{\textcolor{blue}{Google Drive}}
% \newpage

\newpage



\section{\textbf{LỜI MỞ ĐẦU}
}
Trong thời đại công nghệ số phát triển mạnh mẽ, mạng xã hội đã trở thành một phần không thể thiếu trong đời sống con người. Các nền tảng như Facebook, Twitter, Instagram hay Threads của Meta giúp kết nối hàng triệu người dùng trên toàn thế giới, hỗ trợ giao tiếp, chia sẻ thông tin và mở rộng cơ hội kinh doanh.

Threads, một nền tảng mới của Meta, được thiết kế để tối ưu hóa khả năng tương tác nhanh chóng, đơn giản hóa trải nghiệm chia sẻ nội dung dưới dạng văn bản, hình ảnh và video. Với sự thành công ban đầu của Threads, ý tưởng xây dựng một nền tảng mạng xã hội tương tự giúp sinh viên hiểu rõ hơn về cách phát triển một hệ thống mạng xã hội từ đầu, ứng dụng công nghệ hiện đại và tối ưu hóa hiệu suất.

Chính vì vậy, nhóm chúng tôi quyết định thực hiện đồ án “Xây dựng website mạng xã hội Honey” với mục tiêu xây dựng một hệ thống mạng xã hội có đầy đủ các tính năng quan trọng như đăng bài, bình luận, theo dõi người dùng, trò chuyện trực tuyến và thông báo thời gian thực.

\newpage

\section{\textbf{LỜI CAM ĐOAN}
}

Nhóm chúng em xin cam đoan rằng đồ án \textbf{“Xây dựng website mạng xã hội Honey”} là kết quả nghiên cứu và thực hiện của tất cả thành viên, dưới sự hướng dẫn của giảng viên \textbf{Trần Văn Hùng}.

Toàn bộ nội dung trong đồ án này được chúng tôi tìm hiểu, phân tích và triển khai dựa trên kiến thức đã học, các tài liệu tham khảo có trích dẫn đầy đủ. Chúng tôi cam kết không sao chép hoặc sử dụng trái phép nội dung từ bất kỳ nguồn nào mà không ghi rõ nguồn gốc.

Nếu phát hiện có bất kỳ hành vi đạo văn hoặc gian lận nào trong đồ án này, tôi xin chịu hoàn toàn trách nhiệm trước nhà trường và hội đồng đánh giá.

Tôi xin chân thành cảm ơn!

 

\newpage

\section{\textbf{CHƯƠNG 1: TỔNG QUAN VỀ HONEY SOCIAL }}

\subsection{Giới thiệu} 
Trong thời đại công nghệ số phát triển mạnh mẽ, mạng xã hội đã trở thành một phần không thể thiếu trong đời sống con người, hỗ trợ kết nối, chia sẻ thông tin và mở rộng cơ hội kinh doanh. Các nền tảng như Facebook, Twitter, Instagram hay Threads của Meta đã đạt được thành công lớn trong việc kết nối hàng triệu người dùng trên toàn cầu. Tuy nhiên, các nền tảng này vẫn đối mặt với các thách thức như nội dung độc hại, thiếu cá nhân hóa thông minh, và trải nghiệm người dùng chưa tối ưu.

Đồ án ``Xây dựng website mạng xã hội Honey'' được thực hiện với mục tiêu phát triển một nền tảng mạng xã hội hiện đại, tích hợp trí tuệ nhân tạo (AI) để mang lại trải nghiệm an toàn, thông minh và cá nhân hóa cho người dùng. Hệ thống sử dụng các công nghệ tiên tiến như MERN Stack, OpenAI Moderation API, Elasticsearch, và RabbitMQ để đảm bảo hiệu suất cao, khả năng mở rộng, và quản lý nội dung hiệu quả. Chương này sẽ trình bày tổng quan về hệ thống, bao gồm mục đích, yêu cầu chức năng, thiết kế tương tác, phương pháp tiếp cận, và các công nghệ triển khai.

\subsection{Tổng quan hệ thống}
\subsubsection{Mục đích hệ thống}
Mạng xã hội Honey được thiết kế để cung cấp một nền tảng kết nối cộng đồng, cho phép người dùng chia sẻ bài viết, tương tác (thích, bình luận), theo dõi lẫn nhau, và trò chuyện theo thời gian thực. Hệ thống tích hợp AI để cá nhân hóa nội dung, kiểm duyệt tự động, và hỗ trợ người dùng thông qua chatbot thông minh. Mục tiêu chính bao gồm:
\begin{itemize}
    \item Xây dựng một nền tảng mạng xã hội với giao diện thân thiện, responsive, hỗ trợ chế độ sáng/tối.
    \item Tăng cường trải nghiệm người dùng thông qua gợi ý bài viết cá nhân hóa và tìm kiếm thông minh.
    \item Đảm bảo an toàn nội dung bằng cách tích hợp OpenAI Moderation API để kiểm duyệt văn bản và hình ảnh.
    \item Tối ưu hóa hiệu suất hệ thống với Redis caching, RabbitMQ, và Elasticsearch KNN.
\end{itemize}

Hệ thống không nhằm cạnh tranh với các nền tảng lớn như Facebook hay Instagram, mà tập trung vào việc cung cấp một giải pháp mạng xã hội quy mô vừa phải, ứng dụng công nghệ hiện đại để giải quyết các vấn đề về nội dung, bảo mật, và trải nghiệm người dùng.

\subsubsection{Khảo sát các sản phẩm tương tự}
Các nền tảng mạng xã hội hiện tại như Threads, Instagram, và Twitter đã đạt được nhiều thành công, nhưng vẫn tồn tại một số hạn chế:
\begin{itemize}
    \item \textbf{Threads (Meta)}: Tối ưu hóa tương tác nhanh với nội dung dạng văn bản, hình ảnh, và video, nhưng thiếu các tính năng cá nhân hóa thông minh và kiểm duyệt tự động hiệu quả.
    \item \textbf{Instagram}: Tập trung vào hình ảnh và video, nhưng các gợi ý nội dung đôi khi thiếu chính xác và không hỗ trợ chatbot thông minh.
    \item \textbf{Twitter}: Hỗ trợ chia sẻ nhanh chóng, nhưng gặp vấn đề về nội dung độc hại và kiểm duyệt thủ công tốn kém.
\end{itemize}

Dựa trên khảo sát, Honey Social đề xuất các cải tiến:
\begin{itemize}
    \item Tích hợp OpenAI Moderation API để tự động kiểm duyệt nội dung.
    \item Sử dụng vector embeddings và Elasticsearch KNN để gợi ý bài viết cá nhân hóa.
    \item Triển khai chatbot AI hỗ trợ người dùng tương tác và báo cáo nội dung vi phạm.
\end{itemize}

\subsubsection{Yêu cầu hoạt động của ứng dụng}
\subsubsubsection{Phần dành cho người dùng cuối}
Người dùng cuối (end-user) có thể thực hiện các chức năng sau:
\begin{itemize}
    \item \textbf{Đăng ký và đăng nhập}: Tạo tài khoản với xác thực email qua Resend API, đăng nhập an toàn bằng JWT.
    \item \textbf{Quản lý hồ sơ}: Cập nhật thông tin cá nhân, ảnh đại diện (lưu trữ trên Cloudinary), và xem hồ sơ của người dùng khác.
    \item \textbf{Đăng bài viết}: Tạo bài viết với văn bản và hình ảnh, chia sẻ liên kết bài viết.
    \item \textbf{Tương tác bài viết}: Thích, bình luận, và phản hồi bình luận.
    \item \textbf{Xem bảng tin}: Hiển thị bài viết từ người dùng đang theo dõi và gợi ý bài viết dựa trên sở thích.
    \item \textbf{Tìm kiếm nâng cao}: Sử dụng Elasticsearch để tìm kiếm bài viết với fuzzy matching và gợi ý vector.
    \item \textbf{Trò chuyện AI}: Tương tác với chatbot AI để nhận hỗ trợ hoặc gợi ý nội dung.
    \item \textbf{Báo cáo bài viết}: Gửi báo cáo về nội dung vi phạm với lý do cụ thể.
    \item \textbf{Thông báo}: Nhận thông báo thời gian thực về tương tác (thích, bình luận) qua Socket.IO.
\end{itemize}

\subsubsubsection{Phần dành cho người quản trị}
Quản trị viên (admin) có các chức năng:
\begin{itemize}
    \item \textbf{Quản lý người dùng}: Xem, chỉnh sửa, hoặc khóa tài khoản người dùng.
    \item \textbf{Quản lý nội dung báo cáo}: Xem xét và xử lý báo cáo vi phạm (xóa bài viết, khóa tài khoản, hoặc bỏ qua).
    \item \textbf{Phân loại vi phạm}: Đánh giá mức độ vi phạm (nhẹ, vừa, nặng) để đưa ra hành động phù hợp.
    \item \textbf{Giao diện quản trị}: Sử dụng admin dashboard để quản lý báo cáo, tìm kiếm và lọc theo thời gian hoặc mức độ vi phạm.
\end{itemize}

\subsection{Thiết kế tương tác}
Hệ thống Honey Social được thiết kế với giao diện thân thiện, responsive, hoạt động tốt trên cả desktop và mobile. Giao diện hỗ trợ chế độ sáng/tối để tối ưu trải nghiệm người dùng. Các thành phần chính bao gồm:
\begin{itemize}
    \item \textbf{Trang chủ/Bảng tin}: Hiển thị bài viết từ người dùng đang theo dõi và bài viết gợi ý, sử dụng lazy loading để tối ưu tốc độ tải.
    \item \textbf{Hồ sơ người dùng}: Hiển thị thông tin cá nhân, bài viết, số lượng người theo dõi/đang theo dõi, và các nút tương tác (theo dõi, nhắn tin).
    \item \textbf{Giao diện chat}: Hỗ trợ trò chuyện thời gian thực với người dùng khác và chatbot AI, tích hợp Socket.IO.
    \item \textbf{Admin dashboard}: Cung cấp bảng điều khiển để quản lý báo cáo và người dùng, với khả năng tìm kiếm và lọc dữ liệu.
\end{itemize}

\subsection{Phương pháp tiếp cận và giải quyết vấn đề}
\subsubsection{Mô hình tổng quát hệ thống}
Hệ thống Honey Social được xây dựng dựa trên kiến trúc MERN Stack, kết hợp với các công nghệ bổ trợ như OpenAI API, Cloudinary, Redis, và RabbitMQ. Mô hình tổng quát bao gồm:
\begin{itemize}
    \item \textbf{Client-side}: React.js quản lý giao diện, Socket.IO xử lý tương tác thời gian thực.
    \item \textbf{Server-side}: Node.js và Express.js xử lý API, tích hợp RabbitMQ cho tác vụ bất đồng bộ và Redis cho caching.
    \item \textbf{Cơ sở dữ liệu}: MongoDB lưu trữ dữ liệu người dùng, bài viết, tin nhắn, và báo cáo.
    \item \textbf{Bên thứ ba}: OpenAI API (kiểm duyệt và chatbot), Cloudinary (lưu trữ media), Resend API (gửi email xác thực).
\end{itemize}

\subsubsection{Phương pháp xây dựng phần mềm}
Dự án áp dụng phương pháp phát triển phần mềm Agile, với các giai đoạn:
\begin{itemize}
    \item Phân tích yêu cầu: Xác định các chức năng chính và công nghệ sử dụng.
    \item Thiết kế hệ thống: Xây dựng kiến trúc MERN Stack, thiết kế cơ sở dữ liệu, và tích hợp AI.
    \item Phát triển: Triển khai từng module (quản lý người dùng, bài viết, chat, kiểm duyệt).
    \item Triển khai: Đưa hệ thống lên AWS EC2.
\end{itemize}

\subsubsection{Kiến trúc phần mềm}
Hệ thống sử dụng kiến trúc RESTful API với MERN Stack:
\begin{itemize}
    \item \textbf{Frontend}: React.js với component-based architecture, sử dụng Virtual DOM để render nhanh.
    \item \textbf{Backend}: Express.js quản lý routing và middleware, tích hợp JWT cho xác thực.
    \item \textbf{Database}: MongoDB với schema linh hoạt, hỗ trợ index trên userId để tối ưu truy vấn.
\end{itemize}

\subsubsection{Công nghệ triển khai hệ thống}
\subsubsubsection{Server-Side}
\begin{itemize}
    \item \textbf{MongoDB}: Cơ sở dữ liệu NoSQL lưu trữ dữ liệu dạng JSON/BSON, quản lý người dùng, bài viết, tin nhắn, và báo cáo. Sử dụng MongoDB Atlas trên AWS để đảm bảo khả năng mở rộng.
    \item \textbf{Node.js}: Môi trường chạy JavaScript server-side, xử lý đồng thời nhiều kết nối với mô hình bất đồng bộ.
    \item \textbf{Express.js}: Framework nhẹ, hỗ trợ xây dựng API RESTful, quản lý routing và middleware.
    \item \textbf{RabbitMQ}: Xử lý tác vụ bất đồng bộ, như xếp hàng kiểm duyệt nội dung.
    \item \textbf{Redis}: Caching dữ liệu (Cache-Aside) để giảm tải cơ sở dữ liệu, tối ưu API GetFeedPosts.
\end{itemize}

\subsubsubsection{Client-Side}
\begin{itemize}
    \item \textbf{React.js}: Thư viện frontend xây dựng giao diện responsive, sử dụng component tái sử dụng và Virtual DOM.
    \item \textbf{Socket.IO}: Hỗ trợ chat và thông báo thời gian thực, sử dụng WebSocket để giảm độ trễ.
    \item \textbf{Cloudinary}: Lưu trữ và tối ưu hóa hình ảnh (avatar, bài viết), tự động nén sang WebP và điều chỉnh kích thước.
\end{itemize}

\subsubsubsection{Công nghệ bổ trợ}
\begin{itemize}
    \item \textbf{JWT}: Xác thực người dùng với token mã hóa, đảm bảo an toàn phiên đăng nhập.
    \item \textbf{Elasticsearch}: Hỗ trợ tìm kiếm nâng cao và gợi ý bài viết bằng vector search (KNN).
    \item \textbf{OpenAI API}: Kiểm duyệt nội dung (Moderation API) và cung cấp chatbot thông minh.
    \item \textbf{Resend API}: Gửi email xác thực tài khoản.
\end{itemize}

\subsection{Tổng kết chương}
Chương này đã trình bày tổng quan về hệ thống mạng xã hội Honey, bao gồm mục đích, khảo sát các sản phẩm tương tự, yêu cầu chức năng, thiết kế tương tác, và phương pháp triển khai. Hệ thống sử dụng MERN Stack kết hợp với các công nghệ AI hiện đại để mang lại trải nghiệm an toàn, thông minh, và hiệu quả. Chương tiếp theo sẽ đi sâu vào cơ sở lý thuyết của các công nghệ được sử dụng, bao gồm MERN Stack, JWT, Socket.IO, và OpenAI API.

\newpage

\section{\textbf{CHƯƠNG 2: CƠ SỞ LÝ THUYẾT  }}



\subsection{Giới thiệu}
Chương này trình bày cơ sở lý thuyết của các công nghệ được sử dụng trong đồ án "Xây dựng website mạng xã hội Honey". Các công nghệ chính bao gồm MERN Stack (MongoDB, Express.js, React.js, Node.js), cùng với các công nghệ bổ trợ như JWT, Socket.IO, Cloudinary, Elasticsearch, và OpenAI API. Mỗi công nghệ sẽ được phân tích về khái niệm, kiến trúc, ứng dụng trong hệ thống, lợi ích, và cách tối ưu hóa.

\subsection{Mô hình MVC (Model-View-Controller)}

\textbf{Khái niệm:}
MVC là mô hình kiến trúc phần mềm chia ứng dụng thành ba thành phần chính: Model (Mô hình), View (Giao diện), và Controller (Điều khiển). Mô hình này tách biệt logic nghiệp vụ, giao diện người dùng và xử lý yêu cầu, tạo ra code có tính bảo trì cao và dễ mở rộng.

\textbf{Các thành phần của MVC:}

\textbf{1. Model (Mô hình):}
\begin{itemize}
\item Quản lý dữ liệu và logic nghiệp vụ của ứng dụng
\item Tương tác với cơ sở dữ liệu, validation dữ liệu
\item Không phụ thuộc vào View và Controller
\item Thông báo cho View khi dữ liệu thay đổi
\end{itemize}

\textbf{2. View (Giao diện):}
\begin{itemize}
\item Hiển thị dữ liệu cho người dùng và nhận input
\item Render UI components, xử lý tương tác người dùng
\item Không chứa business logic, chỉ hiển thị dữ liệu
\item Lắng nghe thay đổi từ Model để cập nhật giao diện
\end{itemize}

\textbf{3. Controller (Điều khiển):}
\begin{itemize}
\item Xử lý input từ user, điều phối giữa Model và View
\item Nhận request từ View, gọi Model xử lý
\item Cập nhật View với dữ liệu từ Model
\item Không chứa business logic hay presentation logic
\end{itemize}

\textbf{Luồng hoạt động:}
\begin{enumerate}
\item User tương tác với View (click, submit form)
\item View gửi request đến Controller
\item Controller xử lý request và gọi Model
\item Model thực hiện business logic và truy cập database
\item Model trả kết quả về Controller
\item Controller cập nhật View với dữ liệu mới
\item View hiển thị kết quả cho User
\end{enumerate}

\textbf{Lợi ích của MVC:}
\begin{itemize}
\item \textbf{Separation of Concerns}: Tách biệt rõ ràng các chức năng
\item \textbf{Maintainability}: Dễ sửa đổi và bảo trì từng thành phần
\item \textbf{Testability}: Có thể test riêng từng layer
\item \textbf{Reusability}: Tái sử dụng Model và View cho nhiều mục đích
\item \textbf{Scalability}: Hỗ trợ phát triển team và mở rộng hệ thống
\end{itemize}

\textbf{Ứng dụng trong Honey Social:}
\begin{itemize}
\item \textbf{Model}: Mongoose schemas (User, Post, Comment), business logic
\item \textbf{View}: React components hiển thị UI, JSON responses từ API
\item \textbf{Controller}: Express.js route handlers xử lý HTTP requests
\end{itemize}




\subsection{Công nghệ nền tảng}

\subsubsection{MongoDB}
\textbf{Khái niệm và kiến trúc:}
MongoDB là hệ quản trị cơ sở dữ liệu NoSQL thuộc loại document-based, lưu trữ dữ liệu dưới dạng BSON (Binary JSON). Khác với cơ sở dữ liệu quan hệ truyền thống, MongoDB không yêu cầu schema cố định, cho phép lưu trữ các document có cấu trúc khác nhau trong cùng một collection.

\textbf{Đặc điểm kỹ thuật:}
\begin{itemize}
\item \textbf{Document Structure}: Hỗ trợ nested objects và arrays, phù hợp với dữ liệu phức tạp
\item \textbf{Horizontal Scaling}: Hỗ trợ sharding để phân tán dữ liệu trên nhiều server
\item \textbf{ACID Transactions}: Đảm bảo tính nhất quán dữ liệu trong các thao tác phức tạp
\item \textbf{Aggregation Pipeline}: Xử lý và phân tích dữ liệu mạnh mẽ
\end{itemize}

\textbf{Ứng dụng trong Honey Social:}
\begin{itemize}
\item \textbf{Collections chính}: Users (người dùng), Posts (bài viết), Comments (bình luận), Messages (tin nhắn), Reports (báo cáo), Notifications (thông báo)
\item \textbf{Schema linh hoạt}: Trường \texttt{postVector} trong schema \texttt{Post} lưu vector nhúng từ OpenAI API để gợi ý bài viết thông minh
\item \textbf{Quan hệ dữ liệu}: Sử dụng ObjectId để liên kết giữa các collections, \texttt{populate} để join dữ liệu
\end{itemize}

\textbf{Tối ưu hóa hiệu suất:}
\begin{itemize}
\item \textbf{Indexing Strategy}: 
  \begin{itemize}
  \item Compound index trên \texttt{\{userId: 1, createdAt: -1\}} cho API \texttt{GetFeedPosts}
  \item Text index trên \texttt{content} cho tính năng tìm kiếm bài viết
  \item Sparse index trên \texttt{postVector} cho vector search
  \end{itemize}
\item \textbf{Connection Pooling}: Sử dụng mongoose với maxPoolSize = 10 để tối ưu kết nối
\item \textbf{MongoDB Atlas}: Triển khai trên AWS với auto-scaling và backup tự động
\end{itemize}

\subsubsection{Express.js}
\textbf{Khái niệm và kiến trúc:}
Express.js là web framework tối giản và linh hoạt cho Node.js, xây dựng dựa trên middleware pattern. Framework này cung cấp các tính năng mạnh mẽ để phát triển ứng dụng web và API một cách nhanh chóng.

\textbf{Middleware Architecture:}
\begin{itemize}
\item \textbf{Built-in Middleware}: \texttt{express.json()}, \texttt{express.static()}, \texttt{express.urlencoded()}
\item \textbf{Third-party Middleware}: 
  \begin{itemize}
  \item \texttt{cors}: Cross-Origin Resource Sharing
  \item \texttt{helmet}: Security headers
  \item \texttt{morgan}: HTTP request logging
  \item \texttt{express-rate-limit}: Rate limiting
  \end{itemize}
\item \textbf{Custom Middleware}: Authentication, error handling, request validation
\end{itemize}

\textbf{RESTful API Design:}
\begin{itemize}
\item \textbf{Resource-based URLs}: \texttt{/api/users/:id}, \texttt{/api/posts/:postId/comments}
\item \textbf{HTTP Methods}: GET (lấy dữ liệu), POST (tạo mới), PUT (cập nhật), DELETE (xóa)
\item \textbf{Status Codes}: 200 (thành công), 201 (tạo mới), 400 (lỗi client), 401 (chưa xác thực), 500 (lỗi server)
\end{itemize}

\textbf{Ứng dụng trong Honey Social:}
\begin{itemize}
\item \textbf{Route Handlers}: Xử lý đăng ký/đăng nhập, đăng bài viết, tương tác (like, comment), chat thời gian thực
\item \textbf{Middleware JWT}: Xác thực người dùng và phân quyền truy cập API
\item \textbf{Error Handling}: Centralized error handling với custom error classes
\item \textbf{Validation}: Sử dụng joi hoặc express-validator để validate input
\end{itemize}

\subsubsection{React.js}
\textbf{Khái niệm và kiến trúc:}
React.js là thư viện JavaScript để xây dựng giao diện người dùng, sử dụng component-based architecture và Virtual DOM để tối ưu hiệu suất rendering.

\textbf{Core Concepts:}
\begin{itemize}
\item \textbf{Virtual DOM}: Representation của DOM trong memory, cho phép efficient updates
\item \textbf{Component Lifecycle}: Mounting, updating, unmounting phases
\item \textbf{Unidirectional Data Flow}: Dữ liệu chảy từ parent xuống child components
\item \textbf{React Hooks}: useState, useEffect, useContext, useReducer
\end{itemize}

\textbf{Component Architecture:}
\begin{itemize}
\item \textbf{Functional Components}: Sử dụng hooks thay vì class components
\item \textbf{Higher-Order Components (HOC)}: Tái sử dụng logic giữa các components
\item \textbf{Custom Hooks}: Tách logic phức tạp thành hooks có thể tái sử dụng
\end{itemize}

\textbf{Ứng dụng trong Honey Social:}
\begin{itemize}
\item \textbf{UI Components}: Header, Sidebar, PostCard, CommentSection, ChatBox
\item \textbf{Pages}: Home Feed, Profile, Messages, Admin Dashboard
\item \textbf{State Management}: Context API cho global state, local state cho component-specific data
\item \textbf{Routing}: React Router cho SPA navigation
\item \textbf{Responsive Design}: CSS Modules và Material-UI cho giao diện adaptive
\end{itemize}

\textbf{Performance Optimization:}
\begin{itemize}
\item \textbf{React.memo}: Ngăn re-render không cần thiết
\item \textbf{useMemo \& useCallback}: Memoization cho expensive calculations
\item \textbf{Code Splitting}: Lazy loading với React.lazy() và Suspense
\item \textbf{Bundle Optimization}: Webpack optimization cho production build
\end{itemize}

\subsubsection{Node.js}
\textbf{Khái niệm và kiến trúc:}
Node.js là runtime environment cho phép chạy JavaScript trên server-side, sử dụng V8 JavaScript engine của Google Chrome và mô hình event-driven, non-blocking I/O.

\textbf{Event Loop Architecture:}
\begin{itemize}
\item \textbf{Single-threaded}: Main thread xử lý JavaScript code
\item \textbf{Event Loop}: Quản lý callbacks và async operations
\item \textbf{Thread Pool}: Libuv thread pool cho I/O operations
\item \textbf{Callback Queue}: Queue các callback functions để execute
\end{itemize}

\textbf{Core Modules:}
\begin{itemize}
\item \textbf{HTTP/HTTPS}: Tạo web servers
\item \textbf{File System (fs)}: Thao tác với files
\item \textbf{Path}: Xử lý đường dẫn files
\item \textbf{Crypto}: Mã hóa và hash
\item \textbf{Events}: Event emitter pattern
\end{itemize}

\textbf{Ứng dụng trong Honey Social:}
\begin{itemize}
\item \textbf{Backend Server}: Vận hành Express.js server
\item \textbf{Database Integration}: Kết nối và thao tác với MongoDB
\item \textbf{Third-party APIs}: Tích hợp OpenAI API, Cloudinary, Elasticsearch
\item \textbf{Real-time Features}: Socket.IO cho chat và notifications
\item \textbf{Background Jobs}: Xử lý email, image processing, data analytics
\end{itemize}

\subsection{Công nghệ bổ trợ}

\subsubsection{JWT (JSON Web Token)}
\textbf{Khái niệm và cấu trúc:}
JWT là chuẩn mở (RFC 7519) để truyền thông tin an toàn giữa các bên dưới dạng JSON object. Token bao gồm ba phần được mã hóa base64 và ngăn cách bởi dấu chấm.

\textbf{Cấu trúc Token:}
\begin{itemize}
\item \textbf{Header}: Chứa loại token (JWT) và thuật toán mã hóa (HS256, RS256)
\item \textbf{Payload}: Claims (thông tin user, permissions, expiry time)
\item \textbf{Signature}: Đảm bảo token không bị thay đổi, tạo từ header + payload + secret
\end{itemize}

\textbf{Security Considerations:}
\begin{itemize}
\item \textbf{Token Expiry}: Access token (1 giờ), Refresh token (7 ngày)
\item \textbf{Secret Management}: Sử dụng environment variables, rotation định kỳ
\item \textbf{Token Storage}: HttpOnly cookies vs localStorage trade-offs
\item \textbf{CSRF Protection}: SameSite cookie attribute
\end{itemize}

\textbf{Ứng dụng trong Honey Social:}
\begin{itemize}
\item \textbf{Authentication Flow}: Login → Generate JWT → Store in httpOnly cookie
\item \textbf{Authorization}: Middleware kiểm tra JWT trong header \texttt{Authorization}
\item \textbf{User Context}: Decode JWT để lấy userId cho API calls
\item \textbf{Session Management}: Refresh token mechanism cho long-term sessions
\end{itemize}

\subsubsection{WebSocket và Socket.IO}
\textbf{WebSocket Protocol:}
WebSocket là giao thức giao tiếp full-duplex qua single TCP connection, cho phép real-time communication giữa client và server.

\textbf{Protocol Features:}
\begin{itemize}
\item \textbf{Handshake}: HTTP upgrade request để chuyển sang WebSocket
\item \textbf{Frame Structure}: Text frames, binary frames, control frames
\item \textbf{Connection States}: Connecting, open, closing, closed
\item \textbf{Low Latency}: Giảm overhead so với HTTP polling
\end{itemize}

\textbf{Socket.IO Enhancements:}
\begin{itemize}
\item \textbf{Fallback Mechanisms}: Long-polling khi WebSocket không khả dụng
\item \textbf{Room Management}: Grouping connections cho targeted messaging
\item \textbf{Event-based Communication}: Custom events với structured data
\item \textbf{Auto-reconnection}: Tự động kết nối lại khi mất connection
\end{itemize}

\textbf{Ứng dụng trong Honey Social:}
\begin{itemize}
\item \textbf{Real-time Chat}: Tin nhắn tức thời giữa users
\item \textbf{Live Notifications}: Thông báo comment, like, follow ngay lập tức
\item \textbf{Online Status}: Hiển thị trạng thái online/offline của users
\item \textbf{Namespace Organization}: \texttt{/chat}, \texttt{/notifications} cho separation of concerns
\end{itemize}

\subsubsection{Cloudinary}
\textbf{Khái niệm và kiến trúc:}
Cloudinary là cloud-based service chuyên về quản lý, tối ưu hóa và phân phối media assets (images, videos) qua global CDN network.

\textbf{Image Processing Pipeline:}
\begin{itemize}
\item \textbf{Upload}: Direct upload từ client hoặc server-side với signed URLs
\item \textbf{Transformation}: Real-time resize, crop, format conversion, quality optimization
\item \textbf{Delivery}: CDN distribution với edge caching và geo-optimization
\item \textbf{Storage}: Backup và version control cho media assets
\end{itemize}

\textbf{Optimization Techniques:}
\begin{itemize}
\item \textbf{Responsive Images}: Dynamic resizing dựa trên device và viewport
\item \textbf{Format Optimization}: Auto-convert sang WebP/AVIF cho browsers hỗ trợ
\item \textbf{Progressive Loading}: Progressive JPEG và lazy loading
\item \textbf{Compression}: Intelligent quality adjustment để giảm file size
\end{itemize}

\textbf{Ứng dụng trong Honey Social:}
\begin{itemize}
\item \textbf{Media Storage}: Ảnh bài viết, avatar, cover photos, chat images
\item \textbf{Auto-optimization}: Tự động nén và convert format
\item \textbf{Transformation Presets}: Avatar cropping (200x200), post images (max 1200px width)
\item \textbf{CDN Delivery}: Fast loading từ nearest edge server
\end{itemize}

\subsubsection{Elasticsearch và Vector Search}
\textbf{Elasticsearch Architecture:}
Elasticsearch là distributed search engine xây dựng trên Apache Lucene, hỗ trợ full-text search và analytics.

\textbf{Core Concepts:}
\begin{itemize}
\item \textbf{Inverted Index}: Mapping từ terms đến documents chứa terms đó
\item \textbf{Sharding}: Phân tán data trên multiple nodes
\item \textbf{Replication}: Backup data cho high availability
\item \textbf{Cluster}: Tập hợp các nodes working together
\end{itemize}

\textbf{Vector Search (KNN):}
\begin{itemize}
\item \textbf{Dense Vectors}: High-dimensional representations từ machine learning models
\item \textbf{Similarity Metrics}: Cosine similarity, dot product, euclidean distance
\item \textbf{HNSW Algorithm}: Hierarchical Navigable Small World graphs cho fast approximate search
\item \textbf{Vector Indexing}: Optimized index structures cho vector search
\end{itemize}

\textbf{Ứng dụng trong Honey Social:}
\begin{itemize}
\item \textbf{Content Recommendation}: Gợi ý bài viết dựa trên semantic similarity
\item \textbf{Vector Storage}: Lưu OpenAI embeddings trong \texttt{postVector} field
\item \textbf{KNN Search}: Tìm bài viết tương đồng với cosine similarity
\item \textbf{Performance}: Inference Processor giảm query time xuống 50ms
\end{itemize}

\subsubsection{Caching Strategy}
\textbf{Cache-Aside Pattern:}
Application quản lý cache manually, đọc từ cache trước, nếu cache miss thì query database và update cache.

\textbf{Redis Data Structures:}
\begin{itemize}
\item \textbf{Strings}: Simple key-value pairs cho session data
\item \textbf{Hashes}: Object-like structures cho user profiles
\item \textbf{Lists}: Ordered collections cho recent activities
\item \textbf{Sets}: Unordered unique collections cho followers/following
\item \textbf{Sorted Sets}: Leaderboards và ranking features
\end{itemize}

\textbf{Cache Optimization:}
\begin{itemize}
\item \textbf{TTL Management}: Expire keys tự động (feed cache: 5 phút)
\item \textbf{Cache Invalidation}: Xóa cache khi có data updates
\item \textbf{Cache Warming}: Pre-load popular content vào cache
\item \textbf{Memory Management}: Eviction policies (LRU, LFU)
\end{itemize}

\textbf{Ứng dụng trong Honey Social:}
\begin{itemize}
\item \textbf{Feed Caching}: Cache user feeds để giảm database queries
\item \textbf{Session Storage}: Store user sessions trong Redis
\item \textbf{Rate Limiting}: Track API calls per user/IP
\item \textbf{Popular Content}: Cache trending posts và popular hashtags
\end{itemize}

\subsubsection{Lazy Loading}
\textbf{Khái niệm và implementation:}
Lazy loading là kỹ thuật defer loading của resources cho đến khi thực sự cần thiết, giảm initial load time và bandwidth usage.

\textbf{Implementation Techniques:}
\begin{itemize}
\item \textbf{Intersection Observer API}: Detect khi elements enter viewport
\item \textbf{Pagination}: Load content theo chunks thay vì load all
\item \textbf{Virtual Scrolling}: Render chỉ visible items trong long lists
\item \textbf{Image Lazy Loading}: Load images khi scroll đến
\end{itemize}

\textbf{Ứng dụng trong Honey Social:}
\begin{itemize}
\item \textbf{Feed Posts}: Load 10 posts mỗi lần, load thêm khi scroll
\item \textbf{Comment Loading}: Load comments on-demand khi expand
\item \textbf{Image Loading}: Progressive image loading với placeholders
\item \textbf{User Lists}: Lazy load followers/following lists
\end{itemize}

\subsection{Architecture Patterns}

\subsubsection{Monolithic vs Microservices}
\textbf{Current Monolithic Approach:}
\begin{itemize}
\item \textbf{Single Deployment Unit}: Toàn bộ application trong một codebase
\item \textbf{Shared Database}: Tất cả modules sử dụng chung MongoDB instance
\item \textbf{Internal Communication}: Function calls và shared memory
\item \textbf{Pros}: Đơn giản deploy, testing, debugging
\item \textbf{Cons}: Scaling limitations, technology lock-in
\end{itemize}

\textbf{Potential Microservices Evolution:}
\begin{itemize}
\item \textbf{User Service}: Authentication, profiles, relationships
\item \textbf{Content Service}: Posts, comments, media handling
\item \textbf{Notification Service}: Real-time notifications, email, push
\item \textbf{Search Service}: Elasticsearch integration, recommendations
\item \textbf{Analytics Service}: User behavior tracking, metrics
\end{itemize}

\subsubsection{Event-Driven Architecture}
\textbf{Message Queue Integration:}
Sử dụng RabbitMQ cho asynchronous communication và event processing.

\textbf{Event Types:}
\begin{itemize}
\item \textbf{Domain Events}: UserRegistered, PostCreated, CommentAdded
\item \textbf{Integration Events}: Cross-service communication
\item \textbf{System Events}: ErrorOccurred, PerformanceMetric
\end{itemize}

\textbf{Benefits:}
\begin{itemize}
\item \textbf{Loose Coupling}: Services không phụ thuộc trực tiếp
\item \textbf{Scalability}: Scale individual event processors
\item \textbf{Reliability}: Message persistence và retry mechanisms
\item \textbf{Auditability}: Event sourcing cho complete audit trail
\end{itemize}

\subsection{Tổng kết chương}
Chương này đã trình bày cơ sở lý thuyết chi tiết của các công nghệ sử dụng trong hệ thống Honey Social. MERN Stack cung cấp foundation mạnh mẽ với MongoDB cho flexible data storage, Express.js cho robust API development, React.js cho interactive UI, và Node.js cho high-performance backend. Các công nghệ bổ trợ như JWT, Socket.IO, Cloudinary, Elasticsearch, và Redis giải quyết các yêu cầu chuyên biệt về security, real-time communication, media management, intelligent search, và performance optimization. Architecture patterns được áp dụng đảm bảo hệ thống có thể mở rộng và maintainable. Những kiến thức lý thuyết này tạo nền tảng cho việc triển khai thực tế được mô tả trong chương tiếp theo.

\newpage
\section{\textbf{PHÂN TÍCH VÀ THIẾT KẾ HỆ THỐNG}}

\subsection{Phân Tích Hệ Thống}

\subsubsection{Biểu đồ Use Case}

\textbf{Use Case Tổng quát} \\
Hệ thống "Nền tảng Ví điện tử Sinh viên" bao gồm các tác nhân chính và các chức năng tổng quát như sau:

\begin{itemize}
    \item \textbf{Sinh viên}: Đăng ký, Đăng nhập, Thanh toán NFC, Nạp tiền (MoMo), Xem lịch sử giao dịch, Chat với AI.
    \item \textbf{Quản trị viên}: Đăng nhập, Quản lý sinh viên (CRUD), Xem thống kê Dashboard, Quản lý giao dịch.
    \item \textbf{Hệ thống thanh toán (MoMo)}: Xử lý yêu cầu nạp tiền, Gửi thông báo kết quả (IPN).
    \item \textbf{Hệ thống AI (Gemini)}: Phản hồi câu hỏi, Phân tích chi tiêu.
\end{itemize}

% Placeholder cho hình Use Case Tổng quát
\begin{figure}[H]
    \centering
    % \includegraphics[width=1\textwidth]{image/diagrams/usecase_tongquat.png}
    \fbox{\begin{minipage}{0.8\textwidth}
        \centering
        \vspace{2cm}
        \textbf{[SƠ ĐỒ USE CASE TỔNG QUÁT - MERMAID]} \\
        Sinh viên --> (Thanh toán NFC) \\
        Sinh viên --> (Nạp tiền MoMo) \\
        Quản trị viên --> (Quản lý Dashboard)
        \vspace{2cm}
    \end{minipage}}
    \caption{Sơ đồ Use Case Tổng quát}
    \label{fig:usecase_tongquat}
\end{figure}

\subsubsection{Đặc tả các Use Case chính}

\textbf{1. Thanh toán NFC} \\
Cho phép sinh viên thanh toán tại các điểm chấp nhận (căn tin, thư viện) bằng cách chạm điện thoại vào thiết bị đọc thẻ.

\begin{longtable}{|>{\bfseries}m{4cm}|m{10cm}|}
\caption{Đặc tả Use Case: Thanh toán NFC}
\label{table:usecase-nfc}\\
\hline
Tên Use Case & Thanh toán NFC \\
\hline
Mô tả & Sinh viên sử dụng điện thoại có hỗ trợ NFC để thanh toán hóa đơn. \\
\hline
Tác nhân & Sinh viên \\
\hline
Tiền điều kiện &
\begin{itemize}
    \item Sinh viên đã đăng nhập vào ứng dụng Mobile.
    \item Điện thoại có hỗ trợ NFC và NFC đang bật.
    \item Số dư ví đủ để thanh toán.
\end{itemize} \\
\hline
Hậu điều kiện &
\begin{itemize}
    \item Số dư ví bị trừ tương ứng với số tiền thanh toán.
    \item Giao dịch được lưu vào lịch sử.
    \item Thông báo thành công hiển thị trên thiết bị.
\end{itemize} \\
\hline
Luồng sự kiện chính &
\begin{enumerate}
    \item Sinh viên chọn chức năng "Thanh toán" hoặc đưa điện thoại lại gần đầu đọc.
    \item Ứng dụng phát hiện tín hiệu NFC từ đầu đọc.
    \item Ứng dụng hiển thị thông tin giao dịch (Số tiền, Đơn vị thụ hưởng).
    \item Sinh viên xác nhận thanh toán (có thể yêu cầu PIN/Vân tay).
    \item Hệ thống xử lý trừ tiền và gửi xác nhận.
    \item Thông báo "Thanh toán thành công".
\end{enumerate} \\
\hline
Luồng thay thế &
\begin{itemize}
    \item \textbf{Số dư không đủ}: Thông báo lỗi và gợi ý nạp tiền.
    \item \textbf{Lỗi kết nối NFC}: Yêu cầu thử lại hoặc nhập mã thủ công.
\end{itemize} \\
\hline
\end{longtable}

\textbf{2. Nạp tiền qua MoMo} \\
Quy trình nạp tiền vào ví điện tử sinh viên thông qua cổng thanh toán MoMo.

\begin{longtable}{|>{\bfseries}m{4cm}|m{10cm}|}
\caption{Đặc tả Use Case: Nạp tiền qua MoMo}
\label{table:usecase-momo}\\
\hline
Tên Use Case & Nạp tiền qua MoMo \\
\hline
Mô tả & Sinh viên nạp tiền vào tài khoản ví từ ứng dụng MoMo. \\
\hline
Tác nhân & Sinh viên, Hệ thống MoMo \\
\hline
Tiền điều kiện & Sinh viên đã đăng nhập và liên kết hoặc cài đặt ứng dụng MoMo. \\
\hline
Hậu điều kiện & Số dư ví sinh viên tăng lên tương ứng số tiền nạp. \\
\hline
Luồng sự kiện chính &
\begin{enumerate}
    \item Sinh viên chọn "Nạp tiền" trên màn hình chính.
    \item Nhập số tiền cần nạp và chọn nguồn tiền "MoMo".
    \item Hệ thống chuyển hướng sang ứng dụng MoMo.
    \item Sinh viên xác nhận thanh toán trên MoMo.
    \item MoMo xử lý và gửi thông báo (IPN) về Server.
    \item Server cập nhật số dư và thông báo cho Sinh viên.
\end{enumerate} \\
\hline
Luồng thay thế &
\begin{itemize}
    \item \textbf{Hủy giao dịch}: Sinh viên hủy thanh toán trên MoMo $\rightarrow$ Quay về app, hiển thị "Giao dịch bị hủy".
    \item \textbf{Lỗi mạng}: Thông báo "Vui lòng kiểm tra kết nối".
\end{itemize} \\
\hline
\end{longtable}

\textbf{3. Chat với Trợ lý ảo AI} \\
Sinh viên tương tác với chatbot để hỏi về tài chính cá nhân.

\begin{longtable}{|>{\bfseries}m{4cm}|m{10cm}|}
\caption{Đặc tả Use Case: Chat với AI}
\label{table:usecase-ai}\\
\hline
Tên Use Case & Chat với AI Financial Advisor \\
\hline
Mô tả & Sinh viên hỏi đáp với AI về lịch sử giao dịch và nhận lời khuyên tài chính. \\
\hline
Tác nhân & Sinh viên, Hệ thống AI (Gemini) \\
\hline
Tiền điều kiện & Sinh viên đã đăng nhập. \\
\hline
Hậu điều kiện & Sinh viên nhận được câu trả lời từ AI. \\
\hline
Luồng sự kiện chính &
\begin{enumerate}
    \item Sinh viên mở màn hình Chatbot.
    \item Nhập câu hỏi (ví dụ: "Tháng này tôi tiêu bao nhiêu tiền?").
    \item Hệ thống gửi câu hỏi kèm dữ liệu ngữ cảnh (lịch sử giao dịch ẩn danh) đến Gemini API.
    \item Gemini phân tích và trả về câu trả lời văn bản.
    \item Ứng dụng hiển thị câu trả lời cho Sinh viên.
\end{enumerate} \\
\hline
\end{longtable}

\subsection{Thiết Kế Hệ Thống}

\subsubsection{Biểu đồ Activity (Hoạt động)}

\textbf{Quy trình Thanh toán NFC}

% Placeholder cho hình Activity Diagram NFC
\begin{figure}[H]
    \centering
    \fbox{\begin{minipage}{0.8\textwidth}
        \centering
        \vspace{2cm}
        \textbf{[SƠ ĐỒ ACTIVITY: THANH TOÁN NFC - MERMAID]} \\
        Start --> Detect NFC --> Confirm --> Process --> End
        \vspace{2cm}
    \end{minipage}}
    \caption{Sơ đồ hoạt động chức năng Thanh toán NFC}
    \label{fig:activity_nfc}
\end{figure}

\textbf{Quy trình Nạp tiền MoMo}

% Placeholder cho hình Activity Diagram MoMo
\begin{figure}[H]
    \centering
    \fbox{\begin{minipage}{0.8\textwidth}
        \centering
        \vspace{2cm}
        \textbf{[SƠ ĐỒ ACTIVITY: NẠP TIỀN MOMO - MERMAID]} \\
        Start --> Request Payment --> Open MoMo --> Confirm --> IPN Callback --> End
        \vspace{2cm}
    \end{minipage}}
    \caption{Sơ đồ hoạt động chức năng Nạp tiền MoMo}
    \label{fig:activity_momo}
\end{figure}

\subsubsection{Biểu đồ Sequence (Tuần tự)}

\textbf{Luồng tương tác Chatbot AI}

% Placeholder cho hình Sequence Diagram AI
\begin{figure}[H]
    \centering
    \fbox{\begin{minipage}{0.8\textwidth}
        \centering
        \vspace{2cm}
        \textbf{[SƠ ĐỒ SEQUENCE: CHATBOT AI - MERMAID]} \\
        Student -> App -> Server -> Gemini API
        \vspace{2cm}
    \end{minipage}}
    \caption{Sơ đồ tuần tự chức năng Chat AI}
    \label{fig:sequence_ai}
\end{figure}

\newpage
\section{\textbf{THIẾT KẾ CƠ SỞ DỮ LIỆU}}

\subsection{Thiết kế sơ đồ E-R}
Dựa trên các yêu cầu phân tích, hệ thống bao gồm các thực thể chính: Người dùng (Users), Ví (Wallets), Giao dịch (Transactions) và Phiên chat (ChatSessions). Mối quan hệ giữa chúng được mô tả như sau:
\begin{itemize}
    \item Một \textbf{User} có duy nhất một \textbf{Wallet}.
    \item Một \textbf{Wallet} có thể thực hiện nhiều \textbf{Transactions}.
    \item Một \textbf{User} có thể có nhiều \textbf{ChatSessions} với AI.
\end{itemize}

\subsection{Thiết kế chi tiết Collection (MongoDB)}

\textbf{1. Users (Người dùng)}
Lưu trữ thông tin cá nhân và tài khoản của sinh viên và quản trị viên.

\begin{table}[H]
\centering
\renewcommand{\arraystretch}{1.3}
\begin{tabular}{|p{3cm}|p{3cm}|p{2cm}|p{6cm}|}
\hline
\textbf{Tên trường} & \textbf{Kiểu dữ liệu} & \textbf{Bắt buộc} & \textbf{Mô tả} \\
\hline
\_id & ObjectId & Có & Khóa chính \\
\hline
studentCode & String & Có & Mã số sinh viên (Unique) \\
\hline
fullName & String & Có & Họ và tên đầy đủ \\
\hline
email & String & Có & Email trường cấp \\
\hline
password & String & Có & Mật khẩu đã mã hóa (bcrypt) \\
\hline
phoneNumber & String & Có & Số điện thoại liên lạc \\
\hline
role & String & Có & Vai trò: "STUDENT" hoặc "ADMIN" \\
\hline
isActive & Boolean & Có & Trạng thái kích hoạt tài khoản \\
\hline
createdAt & Date & Có & Thời gian tạo \\
\hline
\end{tabular}
\caption{Cấu trúc Collection Users}
\end{table}

\textbf{2. Wallets (Ví điện tử)}
Lưu trữ thông tin số dư và trạng thái ví của sinh viên.

\begin{table}[H]
\centering
\renewcommand{\arraystretch}{1.3}
\begin{tabular}{|p{3cm}|p{3cm}|p{2cm}|p{6cm}|}
\hline
\textbf{Tên trường} & \textbf{Kiểu dữ liệu} & \textbf{Bắt buộc} & \textbf{Mô tả} \\
\hline
\_id & ObjectId & Có & Khóa chính \\
\hline
userId & ObjectId & Có & Tham chiếu đến Users \\
\hline
balance & Number & Có & Số dư hiện tại (VNĐ) \\
\hline
currency & String & Có & Đơn vị tiền tệ (Default: "VND") \\
\hline
status & String & Có & Trạng thái: "ACTIVE", "LOCKED" \\
\hline
lastTransaction & Date & Không & Thời gian giao dịch gần nhất \\
\hline
updatedAt & Date & Có & Thời gian cập nhật số dư cuối cùng \\
\hline
\end{tabular}
\caption{Cấu trúc Collection Wallets}
\end{table}

\textbf{3. Transactions (Giao dịch)}
Lưu trữ lịch sử nạp tiền và thanh toán.

\begin{table}[H]
\centering
\renewcommand{\arraystretch}{1.3}
\begin{tabular}{|p{3cm}|p{3cm}|p{2cm}|p{6cm}|}
\hline
\textbf{Tên trường} & \textbf{Kiểu dữ liệu} & \textbf{Bắt buộc} & \textbf{Mô tả} \\
\hline
\_id & ObjectId & Có & Khóa chính \\
\hline
walletId & ObjectId & Có & Tham chiếu đến Wallets \\
\hline
type & String & Có & Loại GD: "DEPOSIT\_MOMO", "PAYMENT\_NFC" \\
\hline
amount & Number & Có & Số tiền giao dịch \\
\hline
status & String & Có & Trạng thái: "PENDING", "SUCCESS", "FAILED" \\
\hline
description & String & Có & Nội dung giao dịch \\
\hline
metadata & Object & Không & Dữ liệu chi tiết (Mã GD MoMo, ID thiết bị POS) \\
\hline
createdAt & Date & Có & Thời gian thực hiện \\
\hline
\end{tabular}
\caption{Cấu trúc Collection Transactions}
\end{table}

\textbf{4. ChatSessions (Hội thoại AI)}
Lưu trữ lịch sử trò chuyện giữa sinh viên và trợ lý ảo Gemini để duy trì ngữ cảnh.

\begin{table}[H]
\centering
\renewcommand{\arraystretch}{1.3}
\begin{tabular}{|p{3cm}|p{3cm}|p{2cm}|p{6cm}|}
\hline
\textbf{Tên trường} & \textbf{Kiểu dữ liệu} & \textbf{Bắt buộc} & \textbf{Mô tả} \\
\hline
\_id & ObjectId & Có & Khóa chính \\
\hline
userId & ObjectId & Có & Tham chiếu đến Users \\
\hline
messages & Array & Có & Mảng các tin nhắn (role: user/model, content) \\
\hline
summary & String & Không & Tóm tắt nội dung hội thoại \\
\hline
createdAt & Date & Có & Thời gian bắt đầu \\
\hline
updatedAt & Date & Có & Thời gian tin nhắn cuối cùng \\
\hline
\end{tabular}
\caption{Cấu trúc Collection ChatSessions}
\end{table}

\newpage
\subsection{Mô hình quan hệ (Schema Diagram)}
Hình dưới đây mô tả mối quan hệ tham chiếu giữa các collection trong cơ sở dữ liệu MongoDB.

\begin{figure}[H]
    \centering
    % Placeholder cho sơ đồ ER/Schema
    \fbox{\begin{minipage}{0.9\textwidth}
        \centering
        \vspace{2cm}
        \textbf{[SƠ ĐỒ QUAN HỆ CƠ SỞ DỮ LIỆU - MONGODB SCHEMA]} \\
        User (1) --- (1) Wallet \\
        Wallet (1) --- (n) Transactions \\
        User (1) --- (n) ChatSessions
        \vspace{2cm}
    \end{minipage}}
    \caption{Sơ đồ quan hệ cơ sở dữ liệu}
    \label{fig:db_schema}
\end{figure}


\newpage
\section{\textbf{CHƯƠNG 5: KẾT QUẢ THỰC NGHIỆM}}

\textbf{1. Hệ thống kiểm duyệt nội dung đa phương thức}


\begin{figure}[H]
    \centering
    \includegraphics[width=1\textwidth]{image/thucnghiem/report.png}
    \caption{Hình ảnh Bảng kiểm duyệt nội dung vi phạm}
    \label{fig:bao_cao}
\end{figure}



\textbf{Phân tích thông minh với OpenAI Moderation API}
\begin{itemize}
    \item \textbf{Kiểm duyệt đa phương thức}: Phân tích cả văn bản và hình ảnh trong moderationService.ts
    \item \textbf{Phân loại mức độ nghiêm trọng}: Tự động xác định low, medium, high dựa trên điểm số vi phạm
    \item \textbf{Xử lý thời gian thực}: Kiểm tra ngay khi đăng bài, tạo báo cáo tự động trong moderationController.ts
\end{itemize}

\textbf{Dashboard quản trị viên}
\begin{itemize}
    \item Giao diện quản lý báo cáo vi phạm hoàn chỉnh trong AdminPage.tsx
    \item Workflow phê duyệt/từ chối với lý do chi tiết
    \item Thống kê vi phạm theo thời gian và mức độ
\end{itemize}

\newpage

\textbf{2. Tìm kiếm ngữ nghĩa vector-based}

\begin{figure}[H]
    \centering
    \includegraphics[width=1\textwidth]{image/thucnghiem/ngunghia.png}
    \caption{Hình ảnh Ngữ nghĩa}
    \label{fig:nghu_nghia}
\end{figure}

\begin{itemize}
    \item \textbf{Vector embeddings}: Sử dụng OpenAI text-embedding-3-large trong embeddingService.ts
    \item \textbf{Elasticsearch kNN}: Tìm kiếm vector với độ chính xác cao
    \item \textbf{Tìm kiếm đa trường}: Kết hợp users, posts, comments trong một truy vấn
\end{itemize}

\textbf{Indexing thông minh}
\begin{itemize}
    \item \textbf{Auto-indexing}: Tự động tạo embeddings khi tạo bài viết mới
    \item \textbf{Batch processing}: Script đánh index hàng loạt trong indexAdvancedData.ts
\end{itemize}
\newpage
\textbf{3. Hệ thống nhắn tin thời gian thực}
\begin{figure}[H]
    \centering
    \includegraphics[width=1\textwidth]{image/thucnghiem/hinhve.png}
    \caption{Hình ảnh Nhắn tin}
    \label{fig:nhan_tin}
\end{figure}
\textbf{Tính năng nâng cao}
\begin{itemize}
    \item \textbf{Socket.io real-time}: Tin nhắn tức thời với read receipts trong socket.ts
    \item \textbf{Message pending system}: Bảo vệ quyền riêng tư với workflow chấp nhận tin nhắn
    \item \textbf{Media sharing}: Upload ảnh/file với Cloudinary integration
\end{itemize}

\textbf{Quản lý cuộc trò chuyện}
\begin{itemize}
    \item \textbf{Conversation management}: Giao diện chat với tabs Recent, Following, Pending
    \item \textbf{Auto-follow on accept}: Tự động follow khi chấp nhận tin nhắn
    \item \textbf{File upload}: Hỗ trợ nhiều định dạng với preview
\end{itemize}

\textbf{4. Hệ thống gợi ý thông minh}
\begin{figure}[H]
    \centering
    \includegraphics[width=1\textwidth]{image/thucnghiem/recommend.png}
    \caption{Hình ảnh Gợi ý}
    \label{fig:goi_y}
\end{figure}
\textbf{Gợi ý kết bạn dựa trên social graph}
\begin{itemize}
    \item \textbf{Mutual connections}: Phân tích kết nối chung trong recommendationController.ts
    \item \textbf{Popularity scoring}: Tính điểm uy tín dựa trên followers/following ratio
    \item \textbf{Elasticsearch aggregation}: Query phức tạp để tìm gợi ý tối ưu
\end{itemize}

\textbf{Gợi ý nội dung cá nhân hóa}
\begin{itemize}
    \item \textbf{Vector similarity}: So sánh embedding của bài viết đã thích
    \item \textbf{Weighted combination}: Kết hợp nhiều vector với trọng số
    \item \textbf{Fallback mechanism}: Đảm bảo luôn có gợi ý dù thiếu dữ liệu
\end{itemize}

\textbf{5. Trợ lý AI tích hợp}
\begin{figure}[H]
    \centering
    \includegraphics[width=1\textwidth]{image/thucnghiem/ai.png}
    \caption{Hình ảnh Trí tuệ nhân tạo}
    \label{fig:tri_tue_nhan_tao}
\end{figure}
\textbf{OpenAI Assistant với context}
\begin{itemize}
    \item \textbf{Thread-based conversation}: Lưu trữ lịch sử trò chuyện trong openaiService.ts
    \item \textbf{Context awareness}: AI hiểu thông tin user như bio, posts, likes
    \item \textbf{File search integration}: Tìm kiếm thông tin từ database
\end{itemize}

\textbf{Giao diện chat thông minh}
\begin{itemize}
    \item \textbf{Quick suggestions}: Gợi ý câu hỏi nhanh trong ChatAI.tsx
    \item \textbf{Real-time typing}: Hiệu ứng typing indicator
    \item \textbf{Message history}: Lưu trữ và khôi phục cuộc trò chuyện
\end{itemize}

\textbf{6. Hệ thống xác minh email}
\begin{figure}[H]
    \centering
    \includegraphics[width=1\textwidth]{image/thucnghiem/mail.png}
    \caption{Hình ảnh Xác nhận mail}
    \label{fig:xac_nhan}
\end{figure}
\textbf{Security với UX tối ưu}
\begin{itemize}
    \item \textbf{Rate limiting}: Chống spam với cooldown 45 giây trong verificationController.ts
    \item \textbf{Email masking}: Bảo mật thông tin email
    \item \textbf{Protected routes}: Component bảo vệ yêu cầu xác minh trong EmailProtectedRoute.tsx
\end{itemize}

\textbf{Workflow hoàn chỉnh}
\begin{itemize}
    \item \textbf{Email change flow}: Quy trình đổi email an toàn
    \item \textbf{Queue processing}: Xử lý email bất đồng bộ
    \item \textbf{Modal integration}: Giao diện xác minh liền mạch
\end{itemize}

\textbf{Background processing}
\begin{itemize}
    \item \textbf{Social actions worker}: Xử lý like/follow bất đồng bộ trong socialActionsWorker.ts
    \item \textbf{Queue-based architecture}: Tách biệt logic nặng khỏi response time
    \item \textbf{Error handling}: Retry mechanism và fallback
\end{itemize}
\newpage
\textbf{7. Bảo mật và triển khai}
\begin{figure}[H]
    \centering
    \includegraphics[width=1\textwidth]{image/thucnghiem/Deployment.png}
    \caption{Hình ảnh Deployment}
    \label{fig:deployment}
\end{figure}

\textbf{SSL Certificate tự động}
\begin{itemize}
    \item \textbf{Let's Encrypt SSL}: Certificate miễn phí cho domain honeysocial.click
    \item \textbf{Auto-renewal}: Script tự động gia hạn certificate
    \item \textbf{HTTPS redirect}: Tự động chuyển hướng từ HTTP sang HTTPS
\end{itemize}

\textbf{Deployment hoàn chỉnh}
\begin{itemize}
    \item \textbf{Docker containerization}: Triển khai với docker-compose
    \item \textbf{Nginx reverse proxy}: Load balancing và SSL termination
    \item \textbf{Production monitoring}: Health checks và logging
\end{itemize}

\newpage
\section{\textbf{KẾT LUẬN VÀ HƯỚNG PHÁT TRIỂN}}

\subsection{Kết luận}

Sau quá trình nghiên cứu và thực hiện, đồ án "Nền tảng Ví điện tử Sinh viên" đã hoàn thành các mục tiêu đề ra ban đầu, xây dựng thành công một hệ sinh thái thanh toán số khép kín phục vụ môi trường giáo dục đại học.

\textbf{Các kết quả đạt được bao gồm:}
\begin{enumerate}
    \item \textbf{Về mặt công nghệ}:
    \begin{itemize}
        \item Xây dựng thành công ứng dụng di động đa nền tảng bằng \textbf{Flutter}, hoạt động mượt mà trên Android.
        \item Làm chủ công nghệ \textbf{NFC}, cho phép biến điện thoại thành thẻ thanh toán tiện lợi.
        \item Tích hợp thành công cổng thanh toán \textbf{MoMo} và trí tuệ nhân tạo \textbf{Google Gemini}, mang lại trải nghiệm hiện đại cho sinh viên.
        \item Hệ thống Backend \textbf{Node.js/MongoDB} hoạt động ổn định, xử lý tốt các tác vụ đồng thời.
    \end{itemize}
    
    \item \textbf{Về mặt thực tiễn}:
    \begin{itemize}
        \item Giải quyết được bài toán thanh toán không dùng tiền mặt trong trường học (căn tin, gửi xe, đóng phạt thư viện).
        \item Cung cấp công cụ quản lý tài chính cá nhân hiệu quả cho sinh viên.
        \item Giúp nhà trường quản lý dòng tiền và dữ liệu sinh viên minh bạch, chính xác hơn.
    \end{itemize}
\end{enumerate}

\textbf{Hạn chế tồn tại:}
\begin{itemize}
    \item Tính năng NFC hiện tại hoạt động tốt nhất trên Android; khả năng hỗ trợ iOS còn hạn chế do chính sách bảo mật của Apple (chỉ hỗ trợ đọc, hạn chế giả lập thẻ).
    \item Tốc độ phản hồi của Chatbot đôi khi còn phụ thuộc vào độ trễ của API Google Gemini.
    \item Chưa tích hợp các phương thức xác thực sinh trắc học (Vân tay/FaceID) để tăng cường bảo mật cho giao dịch giá trị lớn.
\end{itemize}

\subsection{Hướng phát triển}

Để đưa sản phẩm vào ứng dụng thực tế quy mô lớn, nhóm đề xuất các hướng phát triển trong tương lai:

\subsubsection{1. Mở rộng tính năng và Nền tảng}
\begin{itemize}
    \item \textbf{Hỗ trợ iOS toàn diện}: Nghiên cứu các giải pháp thay thế như QR Code động hoặc Bluetooth Low Energy (BLE) để phục vụ người dùng iPhone nếu rào cản NFC chưa được gỡ bỏ.
    \item \textbf{Bảo mật sinh trắc học}: Tích hợp xác thực vân tay và nhận diện khuôn mặt khi thực hiện thanh toán để thay thế mã PIN truyền thống.
\end{itemize}

\subsubsection{2. Hệ sinh thái tài chính mở rộng}
\begin{itemize}
    \item \textbf{Thanh toán đa dịch vụ}: Mở rộng liên kết thanh toán học phí, bảo hiểm y tế và các dịch vụ xung quanh trường (nhà trọ, tiệm in ấn).
    \item \textbf{Gamification}: Tích hợp tính năng tích điểm đổi quà, xếp hạng chi tiêu hợp lý để khuyến khích sinh viên sử dụng.
\end{itemize}

\subsubsection{3. Ứng dụng Blockchain}
\begin{itemize}
    \item Nghiên cứu ứng dụng Blockchain (Private Chain) để lưu trữ log giao dịch, đảm bảo tính minh bạch tuyệt đối và không thể sửa đổi, giúp việc đối soát tài chính giữa nhà trường và sinh viên trở nên tin cậy hơn.
\end{itemize}

\vspace{1cm}
Đồ án này là bước khởi đầu quan trọng, minh chứng cho khả năng ứng dụng công nghệ mới vào giải quyết các vấn đề thực tiễn trong môi trường giáo dục số. Nhóm thực hiện hy vọng sản phẩm sẽ tiếp tục được hoàn thiện và có cơ hội triển khai thực tế trong tương lai gần.

\clearpage
\section*{TÀI LIỆU THAM KHẢO}
\addcontentsline{toc}{section}{TÀI LIỆU THAM KHẢO}
\begin{thebibliography}{00}

\bibitem{flutter-doc}
Google Developers, "Flutter Documentation - Build apps for any screen." [Online].\\
Available: \url{https://flutter.dev/docs}

\bibitem{momo-api}
MoMo for Business, "MoMo Payment API Integration." [Online].\\
Available: \url{https://developers.momo.vn/}

\bibitem{google-gemini}
Google AI for Developers, "Gemini API Overview." [Online].\\
Available: \url{https://ai.google.dev/docs}

\bibitem{nodejs-doc}
OpenJS Foundation, "Node.js Documentation." [Online].\\
Available: \url{https://nodejs.org/en/docs/}

\bibitem{mongodb-doc}
MongoDB, Inc., "MongoDB Manual." [Online].\\
Available: \url{https://www.mongodb.com/docs/manual/}

\bibitem{nfc-tech}
NFC Forum, "Near Field Communication Technology Standards." [Online].\\
Available: \url{https://nfc-forum.org/}

\bibitem{jwt-intro}
Auth0, "Introduction to JSON Web Tokens." [Online].\\
Available: \url{https://jwt.io/introduction}

\end{thebibliography}

\end{document}