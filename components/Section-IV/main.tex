\newpage

\section{\textbf{CƠ SỞ LÝ THUYẾT}}

\subsection{Giới thiệu}
Chương này trình bày cơ sở lý thuyết của các công nghệ được sử dụng trong đồ án "Nền tảng Ví điện tử Sinh viên". Hệ thống được xây dựng dựa trên sự kết hợp giữa các công nghệ web và di động hiện đại, đảm bảo hiệu suất cao, khả năng mở rộng và trải nghiệm người dùng tối ưu. Các công nghệ chính bao gồm Node.js và Express.js cho Backend, MongoDB cho cơ sở dữ liệu, Flutter cho Mobile App, và React.js cho Web Portal. Bên cạnh đó, hệ thống tích hợp các dịch vụ bên thứ ba quan trọng như MoMo Payment Gateway cho thanh toán và Google Gemini API cho trí tuệ nhân tạo.

\subsection{Mô hình Client-Server}

\textbf{Khái niệm:}
Mô hình Client-Server (Khách-Chủ) là mô hình kiến trúc mạng máy tính trong đó hệ thống được chia thành hai thành phần chính: Client (Máy khách) gửi yêu cầu và Server (Máy chủ) xử lý yêu cầu và phản hồi kết quả. Đây là mô hình cơ bản cho hầu hết các ứng dụng web và di động hiện đại.

\textbf{Cơ chế hoạt động:}
\begin{enumerate}
    \item \textbf{Client}: (Mobile App hoặc Web Browser) gửi yêu cầu (HTTP Request) đến Server thông qua Internet. Yêu cầu chứa thông tin về hành động người dùng muốn thực hiện (ví dụ: lấy số dư, thanh toán).
    \item \textbf{Server}: (Backend API) nhận yêu cầu, xử lý logic nghiệp vụ, truy xuất dữ liệu từ Database nếu cần, và tạo phản hồi.
    \item \textbf{Database}: Lưu trữ và cung cấp dữ liệu cho Server.
    \item \textbf{Server}: Gửi phản hồi (HTTP Response) trả về cho Client, thường dưới dạng JSON.
    \item \textbf{Client}: Nhận phản hồi và hiển thị kết quả cho người dùng.
\end{enumerate}

\textbf{Ưu điểm:}
\begin{itemize}
    \item \textbf{Tập trung hóa}: Dữ liệu được quản lý tập trung tại Server, dễ dàng bảo mật và sao lưu.
    \item \textbf{Khả năng mở rộng}: Có thể nâng cấp Server hoặc thêm nhiều Server để phục vụ lượng lớn Client.
    \item \textbf{Đa nền tảng}: Một Server có thể phục vụ nhiều loại Client khác nhau (Web, Android, iOS).
\end{itemize}

\subsection{Công nghệ Backend}

\subsubsection{Node.js}
\textbf{Khái niệm:}
Node.js là một môi trường chạy mã JavaScript phía máy chủ (server-side runtime environment), được xây dựng trên V8 JavaScript engine của Google Chrome. Node.js cho phép phát triển các ứng dụng mạng nhanh chóng và dễ mở rộng.

\textbf{Đặc điểm nổi bật:}
\begin{itemize}
    \item \textbf{Non-blocking I/O}: Node.js sử dụng mô hình nhập/xuất không chặn, cho phép xử lý nhiều kết nối đồng thời mà không cần tạo luồng mới cho mỗi yêu cầu, giúp tiết kiệm tài nguyên hệ thống.
    \item \textbf{Event-driven}: Hoạt động dựa trên cơ chế sự kiện, giúp xử lý các tác vụ bất đồng bộ hiệu quả.
    \item \textbf{Single Language}: Sử dụng JavaScript cho cả Client và Server, giúp đồng bộ hóa logic và tăng tốc độ phát triển.
\end{itemize}

\textbf{Ứng dụng trong dự án:}
Node.js đóng vai trò là nền tảng cốt lõi cho Backend Server, xử lý toàn bộ logic nghiệp vụ, kết nối Database và tích hợp các API bên thứ ba.

\subsubsection{Express.js}
\textbf{Khái niệm:}
Express.js là một framework web tối giản và linh hoạt dành cho Node.js, cung cấp bộ tính năng mạnh mẽ để xây dựng các ứng dụng web và mobile.

\textbf{Vai trò:}
\begin{itemize}
    \item \textbf{Routing}: Quản lý các đường dẫn API (Endpoint) để Client gửi yêu cầu (ví dụ: \texttt{/api/users}, \texttt{/api/transactions}).
    \item \textbf{Middleware}: Xử lý các tác vụ trung gian như xác thực người dùng (Authentication), ghi log (Logging), xử lý lỗi (Error Handling) trước khi đến controller chính.
    \item \textbf{Hiệu suất}: Giúp xây dựng RESTful API nhanh chóng và cấu trúc code rõ ràng.
\end{itemize}

\subsection{Công nghệ Cơ sở dữ liệu}

\subsubsection{MongoDB}
\textbf{Khái niệm:}
MongoDB là hệ quản trị cơ sở dữ liệu NoSQL mã nguồn mở, lưu trữ dữ liệu dưới dạng văn bản (Document) theo định dạng JSON/BSON linh hoạt, thay vì dạng bảng (Table) như SQL truyền thống.

\textbf{Lý do lựa chọn:}
\begin{itemize}
    \item \textbf{Schema linh hoạt}: Phù hợp với dữ liệu thay đổi thường xuyên như thông tin người dùng, log giao dịch.
    \item \textbf{Hiệu năng cao}: Tốc độ đọc/ghi nhanh, đặc biệt với lượng dữ liệu lớn.
    \item \textbf{Khả năng mở rộng}: Hỗ trợ Horizontal Scaling (Sharding) dễ dàng.
    \item \textbf{Tương thích tốt}: Dữ liệu JSON của MongoDB tương thích hoàn hảo với JavaScript (Node.js/React/React Native).
\end{itemize}

\textbf{Ứng dụng trong dự án:}
Lưu trữ thông tin sinh viên, ví điện tử, lịch sử giao dịch, log hệ thống và cấu hình.

\subsection{Công nghệ Mobile}

\subsubsection{Flutter}
\textbf{Khái niệm:}
Flutter là bộ công cụ phát triển phần mềm giao diện người dùng (UI toolkit) mã nguồn mở do Google phát triển, cho phép xây dựng ứng dụng biên dịch gốc (natively compiled) cho di động, web và desktop từ một codebase duy nhất.

\textbf{Ngôn ngữ Dart:}
Flutter sử dụng ngôn ngữ lập trình Dart, được thiết kế tối ưu cho phát triển giao diện người dùng với tính năng "Hot Reload" giúp xem thay đổi code ngay lập tức mà không cần khởi động lại ứng dụng.

\textbf{Đặc điểm kỹ thuật:}
\begin{itemize}
    \item \textbf{Widget-based}: Mọi thành phần giao diện trong Flutter đều là Widget, giúp tùy biến cao và nhất quán trên các nền tảng.
    \item \textbf{Hiệu suất Native}: Flutter biên dịch code Dart thành mã máy (ARM/x86), không thông qua cầu nối (bridge) JavaScript như React Native, mang lại hiệu suất gần như ứng dụng gốc.
\end{itemize}

\textbf{Thư viện sử dụng:}
\begin{itemize}
    \item \textbf{\texttt{flutter\_nfc\_kit}}: Thư viện hỗ trợ đọc/ghi thẻ NFC đa nền tảng, là nòng cốt cho tính năng thanh toán chạm.
    \item \textbf{\texttt{riverpod}}: Giải pháp quản lý trạng thái (State Management) hiện đại, an toàn và dễ test.
\end{itemize}

\subsection{Công nghệ Web Frontend}

\subsubsection{React.js}
\textbf{Khái niệm:}
React.js là thư viện JavaScript phổ biến nhất hiện nay để xây dựng giao diện người dùng, được phát triển bởi Facebook.

\textbf{Đặc điểm:}
\begin{itemize}
    \item \textbf{Component-based}: Chia nhỏ giao diện thành các thành phần độc lập, có thể tái sử dụng.
    \item \textbf{Virtual DOM}: Tối ưu hóa việc cập nhật giao diện, giúp ứng dụng chạy mượt mà.
    \item \textbf{Hệ sinh thái phong phú}: Rất nhiều thư viện hỗ trợ như Material-UI (giao diện), React Router (điều hướng), Axios (gọi API).
\end{itemize}

\textbf{Ứng dụng trong dự án:}
Xây dựng trang Dashboard quản trị (Admin Portal) cho nhà trường, giúp quản lý sinh viên và thống kê giao dịch trực quan.

\subsection{Công nghệ Tích hợp}

\subsubsection{MoMo Payment Gateway}
\textbf{Giới thiệu:}
MoMo là ví điện tử hàng đầu tại Việt Nam. Cổng thanh toán MoMo cho phép các ứng dụng bên thứ ba tích hợp để thực hiện thanh toán trực tuyến an toàn.

\textbf{Cơ chế tích hợp (IPN - Instant Payment Notification):}
Hệ thống sử dụng cơ chế IPN để xử lý giao dịch nạp tiền tự động:
\begin{enumerate}
    \item Người dùng tạo yêu cầu nạp tiền trên App.
    \item App gọi API MoMo để lấy link thanh toán (QR Code/Deeplink).
    \item Người dùng thanh toán trên App MoMo.
    \item Server MoMo gửi thông báo kết quả (IPN) về Backend Server của dự án.
    \item Backend Server xác thực chữ ký số (Signature) và cập nhật số dư ví sinh viên.
\end{enumerate}

\subsubsection{Google Gemini API}
\textbf{Giới thiệu:}
Google Gemini là mô hình ngôn ngữ lớn (LLM) đa phương thức tiên tiến của Google. API cho phép tích hợp khả năng hiểu và tạo ngôn ngữ tự nhiên vào ứng dụng.

\textbf{Ứng dụng trong Chatbot Tài chính:}
\begin{itemize}
    \item \textbf{Context-aware}: Chatbot được cung cấp ngữ cảnh là lịch sử giao dịch và số dư hiện tại của sinh viên (dữ liệu được ẩn danh/bảo mật).
    \item \textbf{Tư vấn}: Dựa trên dữ liệu, AI phân tích thói quen chi tiêu và đưa ra lời khuyên tài chính, cảnh báo chi tiêu quá đà hoặc gợi ý tiết kiệm.
    \item \textbf{Tương tác tự nhiên}: Hỗ trợ giao tiếp bằng tiếng Việt tự nhiên, thân thiện.
\end{itemize}

\subsection{Giao thức và Bảo mật}

\subsubsection{RESTful API}
Hệ thống sử dụng kiến trúc REST (Representational State Transfer) để thiết kế API, đảm bảo tính thống nhất và dễ dàng tích hợp giữa Mobile, Web và Backend. Các phương thức HTTP chuẩn (GET, POST, PUT, DELETE) được sử dụng để thực hiện các thao tác CRUD.

\subsubsection{JWT (JSON Web Token)}
Sử dụng JWT cho cơ chế xác thực (Authentication) và phân quyền (Authorization). Khi người dùng đăng nhập thành công, Server cấp một Token mã hóa. Client gửi Token này trong header của mỗi yêu cầu tiếp theo để xác minh danh tính mà không cần gửi lại mật khẩu.

\subsubsection{NFC (Near Field Communication)}
Công nghệ giao tiếp trường gần cho phép hai thiết bị kết nối khi đặt gần nhau (dưới 4cm). Trong dự án, NFC được dùng để mô phỏng thẻ thanh toán: điện thoại đóng vai trò thẻ (Card Emulation) hoặc đầu đọc (Reader) để thực hiện giao dịch nhanh chóng, bảo mật.
