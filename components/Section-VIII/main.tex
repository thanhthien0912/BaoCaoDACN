\newpage
\section{\textbf{KẾT LUẬN VÀ HƯỚNG PHÁT TRIỂN}}

\subsection{Kết luận}

Sau quá trình nghiên cứu và thực hiện, đồ án "Nền tảng Ví điện tử Sinh viên" đã hoàn thành các mục tiêu đề ra ban đầu, xây dựng thành công một hệ sinh thái thanh toán số khép kín phục vụ môi trường giáo dục đại học.

\textbf{Các kết quả đạt được bao gồm:}
\begin{enumerate}
    \item \textbf{Về mặt công nghệ}:
    \begin{itemize}
        \item Xây dựng thành công ứng dụng di động đa nền tảng bằng \textbf{Flutter}, hoạt động mượt mà trên Android.
        \item Làm chủ công nghệ \textbf{NFC}, cho phép biến điện thoại thành thẻ thanh toán tiện lợi.
        \item Tích hợp thành công cổng thanh toán \textbf{MoMo} và trí tuệ nhân tạo \textbf{Google Gemini}, mang lại trải nghiệm hiện đại cho sinh viên.
        \item Hệ thống Backend \textbf{Node.js/MongoDB} hoạt động ổn định, xử lý tốt các tác vụ đồng thời.
    \end{itemize}
    
    \item \textbf{Về mặt thực tiễn}:
    \begin{itemize}
        \item Giải quyết được bài toán thanh toán không dùng tiền mặt trong trường học (căn tin, gửi xe, đóng phạt thư viện).
        \item Cung cấp công cụ quản lý tài chính cá nhân hiệu quả cho sinh viên.
        \item Giúp nhà trường quản lý dòng tiền và dữ liệu sinh viên minh bạch, chính xác hơn.
    \end{itemize}
\end{enumerate}

\textbf{Hạn chế tồn tại:}
\begin{itemize}
    \item Tính năng NFC hiện tại hoạt động tốt nhất trên Android; khả năng hỗ trợ iOS còn hạn chế do chính sách bảo mật của Apple (chỉ hỗ trợ đọc, hạn chế giả lập thẻ).
    \item Tốc độ phản hồi của Chatbot đôi khi còn phụ thuộc vào độ trễ của API Google Gemini.
    \item Chưa tích hợp các phương thức xác thực sinh trắc học (Vân tay/FaceID) để tăng cường bảo mật cho giao dịch giá trị lớn.
\end{itemize}

\subsection{Hướng phát triển}

Để đưa sản phẩm vào ứng dụng thực tế quy mô lớn, nhóm đề xuất các hướng phát triển trong tương lai:

\subsubsection{1. Mở rộng tính năng và Nền tảng}
\begin{itemize}
    \item \textbf{Hỗ trợ iOS toàn diện}: Nghiên cứu các giải pháp thay thế như QR Code động hoặc Bluetooth Low Energy (BLE) để phục vụ người dùng iPhone nếu rào cản NFC chưa được gỡ bỏ.
    \item \textbf{Bảo mật sinh trắc học}: Tích hợp xác thực vân tay và nhận diện khuôn mặt khi thực hiện thanh toán để thay thế mã PIN truyền thống.
\end{itemize}

\subsubsection{2. Hệ sinh thái tài chính mở rộng}
\begin{itemize}
    \item \textbf{Thanh toán đa dịch vụ}: Mở rộng liên kết thanh toán học phí, bảo hiểm y tế và các dịch vụ xung quanh trường (nhà trọ, tiệm in ấn).
    \item \textbf{Gamification}: Tích hợp tính năng tích điểm đổi quà, xếp hạng chi tiêu hợp lý để khuyến khích sinh viên sử dụng.
\end{itemize}

\subsubsection{3. Ứng dụng Blockchain}
\begin{itemize}
    \item Nghiên cứu ứng dụng Blockchain (Private Chain) để lưu trữ log giao dịch, đảm bảo tính minh bạch tuyệt đối và không thể sửa đổi, giúp việc đối soát tài chính giữa nhà trường và sinh viên trở nên tin cậy hơn.
\end{itemize}

\vspace{1cm}
Đồ án này là bước khởi đầu quan trọng, minh chứng cho khả năng ứng dụng công nghệ mới vào giải quyết các vấn đề thực tiễn trong môi trường giáo dục số. Nhóm thực hiện hy vọng sản phẩm sẽ tiếp tục được hoàn thiện và có cơ hội triển khai thực tế trong tương lai gần.
