
\newpage

\section{\textbf{TỔNG QUAN VỀ NỀN TẢNG VÍ ĐIỆN TỬ SINH VIÊN}}

\subsection{Giới thiệu}
Trong môi trường đại học hiện đại, các giao dịch tài chính hàng ngày của sinh viên như thanh toán tại căn tin, cửa hàng tiện lợi, hay đóng các khoản phí nhỏ thường vẫn phụ thuộc nhiều vào tiền mặt. Điều này không chỉ gây bất tiện về mặt thời gian xử lý, khó khăn trong việc quản lý chi tiêu cá nhân mà còn tiềm ẩn rủi ro mất mát. Mặc dù các ví điện tử phổ biến như MoMo hay ZaloPay đã phát triển mạnh, nhưng vẫn thiếu một giải pháp chuyên biệt được tùy biến sâu cho hệ sinh thái của trường đại học, tích hợp chặt chẽ với thẻ sinh viên và các dịch vụ nội bộ.

Đồ án ``Nền tảng Ví điện tử Sinh viên'' được xây dựng nhằm giải quyết các vấn đề trên thông qua một hệ thống thanh toán khép kín, hiện đại. Dự án cung cấp một ứng dụng di động đa nền tảng cho sinh viên và một cổng quản trị web cho nhà trường, tích hợp các công nghệ tiên tiến như NFC cho thanh toán một chạm và AI Chatbot để tư vấn tài chính. Chương này sẽ trình bày tổng quan về bối cảnh, mục tiêu, yêu cầu chức năng và kiến trúc kỹ thuật của hệ thống.

\subsection{Tổng quan hệ thống}
\subsubsection{Mục đích hệ thống}
Hệ thống được thiết kế với mục tiêu cốt lõi là hiện đại hóa trải nghiệm thanh toán trong khuôn viên trường học. Các mục tiêu cụ thể bao gồm:
\begin{itemize}
    \item \textbf{Tiện lợi hóa thanh toán}: Thay thế tiền mặt và thẻ vật lý bằng thanh toán NFC một chạm.
    \item \textbf{Quản lý tài chính thông minh}: Cung cấp cho sinh viên công cụ theo dõi lịch sử giao dịch chi tiết và nhận tư vấn chi tiêu từ trợ lý ảo AI.
    \item \textbf{Tối ưu hóa quản lý}: Cung cấp cho nhà trường công cụ quản lý tập trung các giao dịch, doanh thu và tài khoản sinh viên.
    \item \textbf{An toàn và bảo mật}: Đảm bảo tính toàn vẹn của giao dịch và dữ liệu người dùng thông qua các chuẩn bảo mật hiện đại.
\end{itemize}

\subsubsection{Khảo sát hiện trạng và giải pháp}
Hiện tại, sinh viên thường gặp các vấn đề:
\begin{itemize}
    \item \textbf{Tiền mặt}: Tốn thời gian thối tiền, dễ rơi rớt, khó thống kê chi tiêu.
    \item \textbf{Ví điện tử đại chúng}: Chưa tích hợp sâu với các dịch vụ đặc thù của trường (ví dụ: tích hợp thẻ thư viện, vé xe, giảm giá nội bộ).
\end{itemize}

Giải pháp ``Ví điện tử Sinh viên'' mang lại sự khác biệt:
\begin{itemize}
    \item Tích hợp công nghệ NFC, tận dụng hạ tầng thẻ hoặc điện thoại có sẵn.
    \item Hệ thống nạp/rút tiền linh hoạt, kết nối trực tiếp với cổng thanh toán MoMo.
    \item Chatbot AI (Google Gemini) tích hợp sẵn để hỗ trợ giải đáp thắc mắc và tư vấn tài chính cá nhân hóa.
\end{itemize}

\subsubsection{Yêu cầu hoạt động của ứng dụng}
\textbf{Phần dành cho Sinh viên (Mobile App)}

Ứng dụng di động là cổng giao tiếp chính của sinh viên với hệ thống, bao gồm các chức năng:
\begin{itemize}
    \item \textbf{Quản lý tài khoản}: Đăng ký, đăng nhập bảo mật, cập nhật thông tin cá nhân.
    \item \textbf{Ví điện tử}: Xem số dư thời gian thực, nạp tiền vào ví thông qua cổng MoMo.
    \item \textbf{Thanh toán NFC}: Thực hiện thanh toán không tiếp xúc tại các điểm chấp nhận trong trường.
    \item \textbf{Lịch sử giao dịch}: Xem danh sách chi tiết các giao dịch nạp/rút/thanh toán với bộ lọc thời gian.
    \item \textbf{Trợ lý ảo AI}: Chat với AI để hỏi về số dư, lịch sử chi tiêu hoặc nhận lời khuyên quản lý tài chính.
\end{itemize}

\textbf{Phần dành cho Quản trị viên (Web Portal)}

Cổng thông tin web dành cho ban quản lý nhà trường hoặc bộ phận tài chính:
\begin{itemize}
    \item \textbf{Dashboard}: Xem biểu đồ thống kê tổng quan về dòng tiền, số lượng giao dịch và người dùng mới.
    \item \textbf{Quản lý sinh viên}: Thêm, sửa, xóa hoặc khóa tài khoản sinh viên; xem chi tiết hồ sơ.
    \item \textbf{Quản lý giao dịch}: Tra cứu và kiểm soát toàn bộ lịch sử giao dịch trong hệ thống để đối soát.
\end{itemize}

\subsection{Thiết kế tương tác}
Trải nghiệm người dùng (UX) là ưu tiên hàng đầu trong thiết kế hệ thống:
\begin{itemize}
    \item \textbf{Mobile App (Flutter)}: Giao diện Material Design hiện đại, tối ưu cho thao tác một tay. Các luồng thanh toán được tối giản hóa để hoàn tất trong dưới 3 bước.
    \item \textbf{Web Admin (React.js)}: Giao diện Dashboard trực quan, sử dụng các biểu đồ và bảng dữ liệu dynamic để hỗ trợ ra quyết định nhanh chóng. Responsive design giúp quản trị viên có thể truy cập từ nhiều loại thiết bị.
\end{itemize}

\subsection{Phương pháp tiếp cận và giải quyết vấn đề}
\subsubsection{Mô hình tổng quát hệ thống}
Hệ thống hoạt động theo mô hình Client-Server với kiến trúc vi dịch vụ (Microservices) đơn giản hóa:
\begin{itemize}
    \item \textbf{Mobile Client}: Ứng dụng Android xây dựng bằng Flutter, giao tiếp với Server qua RESTful API.
    \item \textbf{Web Client}: Trang quản trị React.js, giao tiếp với cùng hệ thống API.
    \item \textbf{Backend Server}: Node.js/Express đóng vai trò trung tâm xử lý logic nghiệp vụ.
    \item \textbf{Database}: MongoDB lưu trữ dữ liệu phi cấu trúc (NoSQL) đảm bảo khả năng mở rộng.
\end{itemize}

\subsubsection{Kiến trúc phần mềm và Công nghệ}
Dự án sử dụng bộ công nghệ (Tech Stack) hiện đại và phổ biến:
\begin{itemize}
    \item \textbf{Backend}: Node.js và Express.js cung cấp hiệu năng cao cho các tác vụ I/O. Xác thực người dùng an toàn bằng JWT (JSON Web Token).
    \item \textbf{Database}: MongoDB (triển khai trên MongoDB Atlas) cho phép lưu trữ dữ liệu giao dịch và người dùng linh hoạt.
    \item \textbf{Mobile}: Flutter (Dart) cho phép phát triển ứng dụng đa nền tảng với hiệu suất gần như native. Sử dụng thư viện \texttt{flutter\_nfc\_kit} cho tính năng NFC và \texttt{riverpod} cho quản lý trạng thái.
    \item \textbf{Frontend Web}: React.js kết hợp với Material-UI mang lại giao diện quản trị chuyên nghiệp.
    \item \textbf{Tích hợp bên thứ 3}: 
        \begin{itemize}
            \item \textbf{MoMo API}: Cổng thanh toán để nạp tiền vào ví.
            \item \textbf{Google Gemini API}: Cung cấp trí tuệ nhân tạo cho tính năng Chatbot.
        \end{itemize}
    \item \textbf{Hạ tầng (DevOps)}: Backend được triển khai trên Railway, Frontend trên Vercel, đảm bảo tính sẵn sàng cao (High Availability).
\end{itemize}
