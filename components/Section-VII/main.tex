\newpage
\section{\textbf{KẾT QUẢ THỰC NGHIỆM}}

\subsection{Môi trường và Công cụ thực nghiệm}
\begin{itemize}
    \item \textbf{Thiết bị Mobile}: Redmi K70E.
    \item \textbf{Môi trường giả lập}: BlueStacks.
    \item \textbf{Backend Server}: Node.js v18 chạy trên Railway App.
    \item \textbf{Database}: MongoDB Atlas (Cluster Tier M0).
    \item \textbf{Cổng thanh toán}: Môi trường Sandbox của MoMo Developer.
\end{itemize}

\subsection{Kết quả triển khai Ứng dụng Sinh viên (Mobile App)}

\subsubsection{1. Đăng nhập và Màn hình chính}
Giao diện đăng nhập cho phép sinh viên truy cập bằng Mã số sinh viên và mật khẩu. Màn hình chính (Dashboard) hiển thị trực quan số dư hiện tại, các phím tắt chức năng (Nạp tiền, Thanh toán, Lịch sử) và tin tức từ nhà trường.

\begin{figure}[H]
    \centering
    \begin{subfigure}[b]{0.45\textwidth}
        \centering
        \includegraphics[width=0.8\textwidth]{image/app/1.jpg}
        \caption{Màn hình Đăng nhập}
    \end{subfigure}
    \hfill
    \begin{subfigure}[b]{0.45\textwidth}
        \centering
        \includegraphics[width=0.8\textwidth]{image/app/2.jpg}
        \caption{Màn hình Chính (Dashboard)}
    \end{subfigure}
    \caption{Giao diện người dùng cơ bản}
\end{figure}

\subsubsection{2. Chức năng Nạp tiền qua MoMo}
Sinh viên nhập số tiền cần nạp, ứng dụng sẽ tự động chuyển hướng sang App MoMo để xác nhận thanh toán. Sau khi thành công, số dư được cập nhật tức thì nhờ cơ chế IPN.


\begin{figure}[H]
    \centering
    \includegraphics[width=1.0\textwidth]{image/app/momo.png}
    \caption{Quy trình nạp tiền qua ví MoMo}
\end{figure}


\subsubsection{3. Chức năng Thanh toán NFC}
Sinh viên đưa điện thoại lại gần thiết bị thanh toán (POS giả lập hoặc điện thoại khác đóng vai trò máy POS). Giao dịch được xử lý trong vòng 1-2 giây.

\begin{figure}[H]
    \centering
    \includegraphics[width=0.6\textwidth]{image/app/3.jpg}
    \caption{Mô phỏng thanh toán một chạm NFC}
    \label{fig:nfc_result}
\end{figure}
\newpage
\subsubsection{4. Chatbot AI Tư vấn tài chính}
Giao diện chat tích hợp Google Gemini. AI có thể trả lời các câu hỏi như "Số dư của tôi còn bao nhiêu?", "Tháng này tôi đã nạp bao nhiêu tiền?" dựa trên dữ liệu giao dịch thực tế.

\begin{figure}[H]
    \centering
    \includegraphics[width=0.5\textwidth]{image/app/4.jpg}
    \caption{Tương tác với trợ lý ảo AI}
\end{figure}

\subsection{Kết quả triển khai Web Quản trị (Admin Portal)}

\subsubsection{1. Dashboard Thống kê}
Trang tổng quan dành cho nhà trường hiển thị biểu đồ dòng tiền vào/ra, tổng số lượng giao dịch trong ngày và số lượng người dùng mới.

\begin{figure}[H]
    \centering
    \includegraphics[width=0.9\textwidth]{image/app/dashboard.png}
    \caption{Dashboard quản trị viên}
\end{figure}

\subsubsection{2. Quản lý Giao dịch và Sinh viên}
Quản trị viên có thể xem danh sách toàn bộ sinh viên, tìm kiếm theo mã số, xem lịch sử giao dịch chi tiết để hỗ trợ giải quyết khiếu nại.

